

\documentclass[12pt, captions=nooneline, titlepage, footsepline, headsepline, toc=sectionentrywithdots, listof=entryprefix, bibliography=totoc]{scrartcl}
\usepackage{tocbasic}
\usepackage[ngerman]{babel}
\usepackage[backend=biber, style=authortitle]{biblatex}

%! Das Inhaltsverzeichnis wird an dieser Stelle formatiert.
\RedeclareSectionCommands[tocindent=0pt]{section, subsection, subsubsection}
\RedeclareSectionCommands[tocnumwidth=45pt]{section, subsection, subsubsection}


%! Formatierung aller Verzeichnisse
\renewcaptionname{ngerman}{\refname}{Quellenverzeichnis}
\setuptoc{toc}{totoc}
\setuptoc{lof}{totoc}
\setuptoc{lot}{totoc}
%\setuptoc{bibliography}{totoc}
\renewcommand*\listoflofentryname{\bfseries\figurename}
\BeforeStartingTOC[lof]{\renewcommand*\autodot{\space\space\space\space}}
\addtokomafont{captionlabel}{\bfseries}
\renewcommand*\listoflotentryname{\bfseries\tablename}
\BeforeStartingTOC[lot]{\renewcommand*\autodot{\space\space\space\space}}

%! Latex-Packages, die verwendet werden

\usepackage{amsmath}
\usepackage{amssymb}
\usepackage{amsthm}
\usepackage{tabularx}
\usepackage{setspace} 
\usepackage{booktabs}
\usepackage{svg}
\usepackage{graphicx}
\usepackage{float}
\usepackage[a4paper,lmargin={2.5cm},rmargin={2.5cm},tmargin={2cm},bmargin={2cm}]{geometry}

\usepackage{csquotes}


%! Schriftart
\usepackage{helvet}
\usepackage{microtype}
\renewcommand{\familydefault}{\sfdefault}

\usepackage[hidelinks]{hyperref}
\usepackage{acronym}
\renewcommand*{\aclabelfont}[1]{\acsfont{#1}} %Abkürzungsverzeichnis - Formatierung
\usepackage{listings} %zB zum korrekten anzeigen von BASH-Befehlen
\lstset{ %listings Einstellungen
  breaklines=true,
  postbreak=\mbox{\textcolor{red}{$\hookrightarrow$}\space},
}



%! Caption um Tabellen und Abbildungen richtig zu beschriften
\usepackage{caption}
\captionsetup{labelsep=none}
%\addto\captionsngerman{\renewcommand{\figurename}{Abbildung}}
\renewcommand*{\figureformat}{%
  \figurename~\thefigure%
  \autodot{\space\space\space}
}
\renewcommand*{\tableformat}{%
  \tablename~\thetable%
  \autodot{\space\space\space}
}

%! Formatierung des Literaturverzeichnis
\DeclareFieldFormat{url}{\newline\url{#1}}
\DeclareFieldFormat{urldate}{\addcomma\space\bibstring{urlseen}\space#1}

\DefineBibliographyStrings{german}{%
  urlseen = {Abruf am},
}
\setlength\bibitemsep{\baselineskip}

%! Formatierung der Fußnotenzitate / Literaturverzeichnis
\renewcommand*{\newunitpunct}{\addcomma\space} 

%? Normales Zitat...
\DeclareCiteCommand{\zitat}[\mkbibfootnote]
  {\usebibmacro{prenote}}
  {\usebibmacro{citeindex}%
   %\mkbibbrackets{\usebibmacro{cite}}%
   \setunit{\addnbspace}
   \printnames{labelname}%
   \setunit{\labelnamepunct}
   %\printfield[citetitle]{title}%
   \newunit
   \printfield{year}
   \newunit
   \printfield{pages}}
  {\addsemicolon\space}
  {\usebibmacro{postnote}}

%? Online Zitat
\DeclareCiteCommand{\onlinezitat}[\mkbibfootnote]
  {\usebibmacro{prenote}}
  {\usebibmacro{citeindex}%
   %\mkbibbrackets{\usebibmacro{cite}}%
   \setunit{\addnbspace}
   online:
   \printnames{labelname}%
   \setunit{\labelnamepunct}
   %\printfield[citetitle]{title}%
   \newunit
   \printfield{year}
   \printtext{(}\printfield{urlday}.\printfield{urlmonth}\printtext{.}\printfield{urlyear}\printtext{)}}
   %\printurldate}
  %{\addsemicolon\space}
  {\usebibmacro{postnote}}


\DeclareMultiCiteCommand{\zitate}[\mkbibfootnote]{\footpartcite}{\addsemicolon\space}

\addbibresource{literatur.bib}

%! Kopf- und Fußzeile
\usepackage[automark]{scrlayer-scrpage} 
\pagestyle{scrheadings} 
\clearscrheadings 
\clearscrplain 
\rohead{\headmark} 
\lofoot{} 
\cofoot{} 
\rofoot{\pagemark}
\makeatletter
\usepackage{geometry}
\geometry{a4paper,
          left=25mm,right=25mm,top=20mm,bottom=20mm,
          includehead=false, % Kopfzeile außerhalb des Textkörper, also im Rand
          includefoot=false,
          headheight = \baselineskip,
          headsep = \dimexpr\Gm@tmargin-\headheight-10mm,
          footskip = \dimexpr\Gm@bmargin-10mm,
          %showframe,
          bindingoffset=0mm}
% Kopfzeile 1,0 cm Abstand zum Blattrand
% Fußzeile 1,0 cm Abstand zum Blattrand
\makeatother


%! Versuch von besseren Seitenumbrüchen
\clubpenalty = 10000
\widowpenalty = 10000
\displaywidowpenalty = 10000
\widowpenalties= 3 10000 10000 150

\linespread{1.3}
\newcommand\frontmatter{%
    \cleardoublepage
  %\@mainmatterfalse
  \pagenumbering{Roman}}

\newcommand\mainmatter{%
    \cleardoublepage
 % \@mainmattertrue
  \pagenumbering{arabic}}

\newcommand\backmatter{%
  \if@openright
    \cleardoublepage
  \else
    \clearpage
  \fi
 % \@mainmatterfalse
   }

% ? Ab hier beginnen eigene Befehle um den Umgang zu erleichtern   

\newcommand{\absatz}{\vspace{12pt}\noindent}
\newcommand{\logisch}[1]{$``#1``$}
\newcommand{\bild}[3][1.0]{\begin{figure}[H]
                      \centering
                      \includegraphics[width=#1\columnwidth]{bilder/#2}
                      \caption{#3}
                      \label{fig:#3}
                      \end{figure}}
\vfuzz=10pt
\begin{document}
\rofoot{}
\rohead{}
\lohead{\textnormal{Zustimmung Plagiatsprüfung}}
    \vspace*{2mm}

    \begin{minipage}{0.5\columnwidth}
        \includesvg[width=\columnwidth]{vorlage/bilder/ba-gc-logo}
        % Alternativ mit PNG Logo, falls Inkscape nicht installiert werden, bzw. nicht der PATH-Variable hinzugefügt werden soll.
        %! Die SVG-Version sieht im Druck deutlich besser aus.
        %\includegraphics[width=\columnwidth]{vorlage/bilder/ba-gc-logo}
    \end{minipage}
    \begin{minipage}{0.45\columnwidth}
        \begin{flushright}
            {\small nach 4BA-F.219\\}
        \end{flushright}
    \end{minipage}
    \vspace*{2mm}

    \begin{center}
        \textbf{\huge{Erklärung zur Prüfung wissenschaftlicher Arbeiten}}
    \end{center}

    Die Bewertung wissenschaftlicher Arbeiten erfordert die Prüfung auf Plagiate. Die hierzu von der Staatlichen Studienakademie Glauchau eingesetzte Prüfungskommission nutzt sowohl eigene Software als auch diesbezügliche Leistungen von Drittanbietern. Dies erfolgt gemäß \href{https://www.revosax.sachsen.de/vorschrift/1672-Saechsisches-Datenschutzgesetz#p7}{§ 7 des Gesetzes zum Schutz der informationellen Selbstbestimmung im Freistaat Sachsen (Sächsisches Datenschutzgesetz - SächsDSG)} vom 25. August 2003 (Rechtsbereinigt mit Stand vom 31. Juli 2011) im Sinne einer Datenverarbeitung im Auftrag.

    Der Studierende bevollmächtigt die Mitglieder der Prüfungskommission hiermit zur Inanspruchnahme o. g. Dienste. In begründeten Ausnahmefällen kann der Datenschutzbeauftragte der Berufsakademie Sachsen sowohl vom Verfasser der wissenschaftlichen Arbeit als auch von der Prüfungskommission in den Entscheidungsprozess einbezogen werden.

    \arrayrulewidth=0.5pt
    \begin{table}[H]
        \centering
        \begin{tabularx}{\columnwidth}{|p{3.2cm}|X|}
            \hline
            Name:             & \autoreins\\
            \hline
            Matrikelnummer:   & \matnumeins\\
            \hline
            Studiengang:      & \studiengang\\
            \hline
            Titel der Arbeit: & \titel\\
            \hline
            Datum:            & \abgabedatum\\
            \hline
            Unterschrift:     & \\
                              & \\
            \hline
        \end{tabularx}
    \end{table}

    \vfill
\end{document}
