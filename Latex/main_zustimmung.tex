%! LaTeX Vorlage
\documentclass[12pt, fleqn, captions=nooneline, titlepage, footsepline, headsepline, toc=chapterentrywithdots, listof=entryprefix, bibliography=totoc, parskip=half-]{scrreprt}

\usepackage{silence} %unnötige Warnungen unterdrücken
\WarningFilter{latex}{You have requested}
% \WarningFilter{scrlayer-scrpage}{\headheight to low}
% \WarningFilter{scrlayer-scrpage}{\footheight to low}
% \WarningFilter{scrlayer-scrpage}{Very small head height detected}
% \WarningFilter{fvextra}{} % was caused by loading csquotes before minted (which loads fvextra)
% \WarningFilter{lineno}{}

\pdfsuppresswarningpagegroup=1

\ProvidesPackage{metadaten}
\usepackage{metadaten}

\usepackage{tocbasic}
\usepackage[ngerman]{babel}
\usepackage[%
  backend=biber,
  labeldateparts=true,
  style=authortitle,
  isbn=false,
  dashed=false,
  maxnames=3]{biblatex}

%!  Änderungen für scrreprt
\renewcommand{\autodot}{}
\usepackage{chngcntr}
\counterwithout{figure}{chapter}
\counterwithout{table}{chapter}
\counterwithout{footnote}{chapter}
\RedeclareSectionCommand[style=section,afterskip=.15em]{chapter}
\setcounter{secnumdepth}{\subsubsectionnumdepth}
\setcounter{tocdepth}{\subsubsectionnumdepth}
\addtokomafont{chapter}{\LARGE}
\addtokomafont{section}{\Large}
\addtokomafont{subsection}{\large}
\addtokomafont{subsubsection}{\normalsize}
\renewcommand*{\chaptermarkformat}{}

\input{vorlage/vorlage_subs/verzeichnisse}

%! Ermöglicht die Ausgabe des aktuellen Titels.
\usepackage{nameref}
\makeatletter
\newcommand*{\currentname}{\@currentlabelname}
\makeatother

%! zusätzliche LaTeX-Packages
\usepackage[table]{xcolor}
\usepackage{amsmath}
\usepackage{amssymb}
\usepackage{amsthm}
\usepackage{tabularx}
\usepackage{multirow}
\usepackage{booktabs}
\usepackage{svg}
\usepackage{graphicx}
\usepackage{float}
%! floatplacement für Codes muss in minted.tex definiert werden
\floatplacement{figure}{htbp}
\floatplacement{table}{htbp}
\floatplacement{figure}{htbp}
\floatplacement{formel}{htbp}
\usepackage[a4paper,lmargin={2.5cm},rmargin={2.5cm},tmargin={2cm},bmargin={2cm}]{geometry}
\usepackage{lineno}
\usepackage[T1]{fontenc}
\usepackage{listings}
\usepackage{tikz}
\usepackage{varwidth}
\usepackage{ifthen}
\usepackage{etoolbox}

%! Abbildungen von Verzeichnisstrukturen
\input{vorlage/vorlage_subs/dirtree}

%! PageBreaks nach jeder Section
\let\oldchapter\chapter
\renewcommand\chapter{\clearpage\oldchapter}

%! Code-Umgebungen
%! Code-Integration im Dokument
%? Inklusive Erzeugung eines Custom-Enviroments für Programmcodes
\usepackage{minted}

% ! Anpassung der Minted-Umgebung, um Abstände zu vereinheitlichen und die Optik zu verbessern
\let\oldminted\minted
\let\oldendminted\endminted
\def\minted{\begingroup \vspace{-0.3cm} \oldminted}
\def\endminted{\oldendminted \vspace{-0.5cm} \endgroup}

\xapptocmd{\inputminted}{\vspace{-0.5cm}}{}{}

%Zeilennummern neu definieren, um Warnungen zu vermeiden und modern anmutende Zahlen zu nutzen
\renewcommand{\theFancyVerbLine}{\scriptsize{\arabic{FancyVerbLine}}}

%Standardformatierung für minted-Umgebung erstellen
\setminted{
    tabsize=2,
    breaklines,
    autogobble,
    fontfamily=courier,
    linenos,
    %! see https://pygments.org/demo/#try, other nice options: paraiso-light, solarized-light, rainbow_dash, gruvbox-light, stata, tango
    style=emacs,
    fontsize=\footnotesize
}
%Keine Zeilennnummer, wenn der einzeilige \mint-Befehl genutzt wird
\xpretocmd{\mint}{\setminted{linenos=false}}{}{}
\xpretocmd{\minted}{\setminted{linenos=true}}{}{}

\DeclareNewTOC[
  type=code,                           % Name der Umgebung
  types=codes,                         % Erweiterung (\listofschemes)
  float,                               % soll gleiten
  floatpos=htbp,
  tocentryentrynumberformat=\bfseries, % voreingestellte Gleitparameter
  name=Code,                           % Name in Überschriften
  listname={Programmcodeverzeichnis},  % Listenname
  % counterwithin=chapter
]{loc}
\setuptoc{loc}{totoc}

\renewcommand*{\codeformat}{%
  \codename~\thecode%
  \autodot{\space\space\space}
}
\BeforeStartingTOC[loc]{\renewcommand*\autodot{\space\space\space\space}}
\AtBeginEnvironment{minted}{\vspace{\baselineskip}}


\usepackage{scrhack} %necessary to use float with scrreprt
\usepackage{csquotes}

%! Schriftart
%! Die HAWA schreibt Arial vor, welches in Standard LaTeX-Distributionen nicht mitgeliefert wird. Helvetica ist nahezu identisch.
\usepackage{helvet}
\usepackage{microtype}
\renewcommand{\familydefault}{\sfdefault}
%! Hyperref und PDF-meta
\usepackage[hidelinks]{hyperref}

%! PDF-Metadaten (Autor/Titel/Beschreibung)
\input{vorlage/vorlage_subs/pdf_metadaten}
\usepackage[numbered]{bookmark}
\usepackage[printonlyused]{acronym}
\usepackage{enumitem}

%Abkürzungsverzeichnis - Formatierung (bspw. zur korrekten Anzeige von BASH-Befehlen)
\renewcommand*{\aclabelfont}[1]{\acsfont{#1}}

%! Untertitel/Legenden
\input{vorlage/vorlage_subs/captions}

%! Literaturverzeichnis und Zitierbefehle
\input{vorlage/vorlage_subs/literatur_zitate}

%! Kopf- und Fußzeilen
\input{vorlage/vorlage_subs/kopf_fußzeile}

%! verbesserte Umbrüche (hoffentlich)
\input{vorlage/vorlage_subs/umbrüche}

%? wird am Ende geladen, um einheitliche Spacings sicherzustellen
\usepackage{setspace}
%! setzt den Zeilenabstand in Floats (Tabellen, Code-Listings usw.) auf den in LaTeX definierten Wert (statt 1.0)
\makeatletter
\let\@xfloat=\latex@xfloat
\makeatother

\vfuzz=10pt
\begin{document}
\rofoot{}
\rohead{}
\lohead{\textnormal{Zustimmung Plagiatsprüfung}}
    \vspace*{2mm}

    \begin{minipage}{0.5\columnwidth}
        \includesvg[width=\columnwidth]{vorlage/bilder/ba-gc-logo}
        % Alternativ mit PNG Logo, falls Inkscape nicht installiert werden, bzw. nicht der PATH-Variable hinzugefügt werden soll.
        %! Die SVG-Version sieht im Druck deutlich besser aus.
        %\includegraphics[width=\columnwidth]{vorlage/bilder/ba-gc-logo}
    \end{minipage}
    \begin{minipage}{0.45\columnwidth}
        \begin{flushright}
            {\small nach 4BA-F.219\\}
        \end{flushright}
    \end{minipage}
    \vspace*{2mm}

    \begin{center}
        \textbf{\huge{Erklärung zur Prüfung wissenschaftlicher Arbeiten}}
    \end{center}

    Die Bewertung wissenschaftlicher Arbeiten erfordert die Prüfung auf Plagiate. Die hierzu von der Staatlichen Studienakademie Glauchau eingesetzte Prüfungskommission nutzt sowohl eigene Software als auch diesbezügliche Leistungen von Drittanbietern. Dies erfolgt gemäß \href{https://www.revosax.sachsen.de/vorschrift/1672-Saechsisches-Datenschutzgesetz#p7}{§ 7 des Gesetzes zum Schutz der informationellen Selbstbestimmung im Freistaat Sachsen (Sächsisches Datenschutzgesetz - SächsDSG)} vom 25. August 2003 (Rechtsbereinigt mit Stand vom 31. Juli 2011) im Sinne einer Datenverarbeitung im Auftrag.

    \ifthenelse{\isundefined{\autorzwei}}{Der Studierende bevollmächtigt}{Die Studierenden bevollmächtigen} die Mitglieder der Prüfungskommission hiermit zur Inanspruchnahme o. g. Dienste. In begründeten Ausnahmefällen kann der Datenschutzbeauftragte der Berufsakademie Sachsen sowohl \ifthenelse{\isundefined{\autorzwei}}{vom Verfasser}{von den Verfassern} der wissenschaftlichen Arbeit als auch von der Prüfungskommission in den Entscheidungsprozess einbezogen werden.

    \arrayrulewidth=0.5pt
    % Tabellen für Unterschriften
    \ifthenelse{\isundefined{\autorzwei}}{\begin{table}[H]
    \centering
    \begin{tabularx}{\columnwidth}{|p{3.2cm}|X|}
        \hline
        Name:             & \autoreins\\
        \hline
        Matrikelnummer:   & \matnumeins\\
        \hline
        Studiengang:      & \studiengang\\
        \hline
        Titel der Arbeit: & \titel\\
        \hline
        Datum:            & \abgabedatum\\
        \hline
        Unterschrift:     & \\
                          & \\
        \hline
    \end{tabularx}
\end{table}

\vfill}{%
    \ifthenelse{\isundefined{\autordrei}}{\begin{table}[H]
    \centering
    \newcolumntype{Y}{>{\centering\arraybackslash}X}
    \begin{tabularx}{\columnwidth}{|X|Y|Y|}
        \hline
        Namen:            & \autoreins  & \autorzwei  \\
        \hline
        Matrikelnummern:  & \matnumeins & \matnumzwei \\
        \hline
        Studiengang:      & \multicolumn{2}{c|}{\studiengang}\\
        \hline
        Titel der Arbeit: & \multicolumn{2}{c|}{\titel}\\
                          %& \multicolumn{2}{c|}{zweite Zeile - falls nötig, \titel durch ersten Teil des Titels ersetzen und hier Rest einfügen}\\
        \hline
        Datum:            & \multicolumn{2}{c|}{\abgabedatum}\\
        \hline
        Unterschriften:   &             &\\
                          &             &\\
        \hline
    \end{tabularx}
\end{table}

\vfill
}{%
        \ifthenelse{\isundefined{\autorvier}}{\begin{table}[H]
    \centering
    \newcolumntype{Y}{>{\centering\arraybackslash}X}
    \begin{tabularx}{\columnwidth}{|X|Y|Y|Y|}
        \hline
        Namen:            & \autoreins  & \autorzwei  & \autordrei \\
        \hline
        Matrikelnummern:  & \matnumeins & \matnumzwei & \matnumdrei \\
        \hline
        Studiengang:      & \multicolumn{3}{c|}{\studiengang}\\
        \hline
        Titel der Arbeit: & \multicolumn{3}{c|}{\titel}\\
                          %& \multicolumn{2}{c|}{zweite Zeile - falls nötig, \titel durch ersten Teil des Titels ersetzen und hier Rest einfügen}\\
        \hline
        Datum:            & \multicolumn{3}{c|}{\abgabedatum}\\
        \hline
        Unterschriften:   &             &             & \\
                          &             &             &\\
        \hline
    \end{tabularx}
\end{table}

\vfill
}{\begin{table}[H]
    \centering
    \newcolumntype{Y}{>{\centering\arraybackslash}X}
    \begin{tabularx}{\columnwidth}{|X|Y|Y|Y|Y|}
        \hline
        Namen:            & \autoreins  & \autorzwei  & \autordrei  & \autorvier\\
        \hline
        Mat.-Num.:        & \matnumeins & \matnumzwei & \matnumdrei & \matnumvier\\
        \hline
        Studiengang:      & \multicolumn{4}{c|}{\studiengang}\\
        \hline
        Titel der Arbeit: & \multicolumn{4}{c|}{\titel}\\
                          %& \multicolumn{2}{c|}{zweite Zeile - falls nötig, \titel durch ersten Teil des Titels ersetzen und hier Rest einfügen}\\
        \hline
        Datum:            & \multicolumn{4}{c|}{\abgabedatum}\\
        \hline
        Unterschriften:   &              &             &            &\\
                          &              &             &            &\\
        \hline
    \end{tabularx}
\end{table}

\vfill
}
    }
    }

\end{document}
