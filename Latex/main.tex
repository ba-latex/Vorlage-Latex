% main.tex - Hauptdatei der Vorlage
% Vorlage


\documentclass[12pt, captions=nooneline, titlepage, footsepline, headsepline, toc=sectionentrywithdots, listof=entryprefix, bibliography=totoc]{scrartcl}
\usepackage{tocbasic}
\usepackage[ngerman]{babel}
\usepackage[backend=biber, style=authortitle]{biblatex}

%! Das Inhaltsverzeichnis wird an dieser Stelle formatiert.
\RedeclareSectionCommands[tocindent=0pt]{section, subsection, subsubsection}
\RedeclareSectionCommands[tocnumwidth=45pt]{section, subsection, subsubsection}


%! Formatierung aller Verzeichnisse
\renewcaptionname{ngerman}{\refname}{Quellenverzeichnis}
\setuptoc{toc}{totoc}
\setuptoc{lof}{totoc}
\setuptoc{lot}{totoc}
%\setuptoc{bibliography}{totoc}
\renewcommand*\listoflofentryname{\bfseries\figurename}
\BeforeStartingTOC[lof]{\renewcommand*\autodot{\space\space\space\space}}
\addtokomafont{captionlabel}{\bfseries}
\renewcommand*\listoflotentryname{\bfseries\tablename}
\BeforeStartingTOC[lot]{\renewcommand*\autodot{\space\space\space\space}}

%! Latex-Packages, die verwendet werden

\usepackage{amsmath}
\usepackage{amssymb}
\usepackage{amsthm}
\usepackage{tabularx}
\usepackage{setspace} 
\usepackage{booktabs}
\usepackage{svg}
\usepackage{graphicx}
\usepackage{float}
\usepackage[a4paper,lmargin={2.5cm},rmargin={2.5cm},tmargin={2cm},bmargin={2cm}]{geometry}

\usepackage{csquotes}


%! Schriftart
\usepackage{helvet}
\usepackage{microtype}
\renewcommand{\familydefault}{\sfdefault}

\usepackage[hidelinks]{hyperref}
\usepackage{acronym}
\renewcommand*{\aclabelfont}[1]{\acsfont{#1}} %Abkürzungsverzeichnis - Formatierung
\usepackage{listings} %zB zum korrekten anzeigen von BASH-Befehlen
\lstset{ %listings Einstellungen
  breaklines=true,
  postbreak=\mbox{\textcolor{red}{$\hookrightarrow$}\space},
}



%! Caption um Tabellen und Abbildungen richtig zu beschriften
\usepackage{caption}
\captionsetup{labelsep=none}
%\addto\captionsngerman{\renewcommand{\figurename}{Abbildung}}
\renewcommand*{\figureformat}{%
  \figurename~\thefigure%
  \autodot{\space\space\space}
}
\renewcommand*{\tableformat}{%
  \tablename~\thetable%
  \autodot{\space\space\space}
}

%! Formatierung des Literaturverzeichnis
\DeclareFieldFormat{url}{\newline\url{#1}}
\DeclareFieldFormat{urldate}{\addcomma\space\bibstring{urlseen}\space#1}

\DefineBibliographyStrings{german}{%
  urlseen = {Abruf am},
}
\setlength\bibitemsep{\baselineskip}

%! Formatierung der Fußnotenzitate / Literaturverzeichnis
\renewcommand*{\newunitpunct}{\addcomma\space} 

%? Normales Zitat...
\DeclareCiteCommand{\zitat}[\mkbibfootnote]
  {\usebibmacro{prenote}}
  {\usebibmacro{citeindex}%
   %\mkbibbrackets{\usebibmacro{cite}}%
   \setunit{\addnbspace}
   \printnames{labelname}%
   \setunit{\labelnamepunct}
   %\printfield[citetitle]{title}%
   \newunit
   \printfield{year}
   \newunit
   \printfield{pages}}
  {\addsemicolon\space}
  {\usebibmacro{postnote}}

%? Online Zitat
\DeclareCiteCommand{\onlinezitat}[\mkbibfootnote]
  {\usebibmacro{prenote}}
  {\usebibmacro{citeindex}%
   %\mkbibbrackets{\usebibmacro{cite}}%
   \setunit{\addnbspace}
   online:
   \printnames{labelname}%
   \setunit{\labelnamepunct}
   %\printfield[citetitle]{title}%
   \newunit
   \printfield{year}
   \printtext{(}\printfield{urlday}.\printfield{urlmonth}\printtext{.}\printfield{urlyear}\printtext{)}}
   %\printurldate}
  %{\addsemicolon\space}
  {\usebibmacro{postnote}}


\DeclareMultiCiteCommand{\zitate}[\mkbibfootnote]{\footpartcite}{\addsemicolon\space}

\addbibresource{literatur.bib}

%! Kopf- und Fußzeile
\usepackage[automark]{scrlayer-scrpage} 
\pagestyle{scrheadings} 
\clearscrheadings 
\clearscrplain 
\rohead{\headmark} 
\lofoot{} 
\cofoot{} 
\rofoot{\pagemark}
\makeatletter
\usepackage{geometry}
\geometry{a4paper,
          left=25mm,right=25mm,top=20mm,bottom=20mm,
          includehead=false, % Kopfzeile außerhalb des Textkörper, also im Rand
          includefoot=false,
          headheight = \baselineskip,
          headsep = \dimexpr\Gm@tmargin-\headheight-10mm,
          footskip = \dimexpr\Gm@bmargin-10mm,
          %showframe,
          bindingoffset=0mm}
% Kopfzeile 1,0 cm Abstand zum Blattrand
% Fußzeile 1,0 cm Abstand zum Blattrand
\makeatother


%! Versuch von besseren Seitenumbrüchen
\clubpenalty = 10000
\widowpenalty = 10000
\displaywidowpenalty = 10000
\widowpenalties= 3 10000 10000 150

\linespread{1.3}
\newcommand\frontmatter{%
    \cleardoublepage
  %\@mainmatterfalse
  \pagenumbering{Roman}}

\newcommand\mainmatter{%
    \cleardoublepage
 % \@mainmattertrue
  \pagenumbering{arabic}}

\newcommand\backmatter{%
  \if@openright
    \cleardoublepage
  \else
    \clearpage
  \fi
 % \@mainmatterfalse
   }

% ? Ab hier beginnen eigene Befehle um den Umgang zu erleichtern   

\newcommand{\absatz}{\vspace{12pt}\noindent}
\newcommand{\logisch}[1]{$``#1``$}
\newcommand{\bild}[3][1.0]{\begin{figure}[H]
                      \centering
                      \includegraphics[width=#1\columnwidth]{bilder/#2}
                      \caption{#3}
                      \label{fig:#3}
                      \end{figure}}
% vordefinierte Kommandos der Vorlage
%! Eigene Befehle zur erleichterten Nutzung
% Hilfsbefehle
\newcommand{\fontheightsvg}[1]{\includesvg[height=1.75ex, inkscapelatex=false]{#1}}
\newcommand{\dtfolder}{\fontheightsvg{vorlage/dirtree_folder}\hspace{0.1cm}}
\newcommand{\dtfile}{\fontheightsvg{vorlage/dirtree_file}\hspace{0.1cm}}

% Umgebungen u.Ä.
\newcommand{\fn}[1]{\footnote{\hspace{0.5em}#1}}
\newcommand{\bild}[4][1.0]{\begin{figure}[H]
  \centering
  \includegraphics[width=#1\columnwidth]{bilder/#2}
  \caption{#3}
  \label{#4}
  \end{figure}}
\newcommand{\striche}[1]{\glqq #1\grqq{}}
\newcommand{\svg}[4][1.0]{\begin{figure}[H]
    \centering
    \includesvg[width=#1\columnwidth,inkscapelatex=false]{bilder/#2}
    \caption{#3}
    \label{#4}
    \end{figure}}
\newcommand{\verzeichnis}[3]{\begin{figure}[H]
  % https://tex.stackexchange.com/a/99591/220899
  \renewcommand{\DTstyle}{\textrm\expandafter\raisebox{-0.7ex}}
  \centering
  \begin{varwidth}{\textwidth}
    \dirtree{#1}  
  \end{varwidth}
  \caption{#2}
  \label{#3}
  \end{figure}}
\newcommand{\logisch}[1]{$``#1``$}
\newcommand{\vglink}[2]{\footnote{\hspace{0.5em}vgl.~\href{#1}{#1}~(#2)}}
\newcommand{\python}[1]{\mintinline{python}{#1}}

% Referenzierung
\newcommand{\uniliteref}[2]{\emph{\hyperref[{#2}]{#1 \ref{#2} - \nameref{#2}}}}
\newcommand{\unifullref}[2]{(\emph{\hyperref[{#2}]{siehe #1 \ref{#2} - \nameref{#2}}})}

% Anhänge
\newcommand{\litearef}[1]{\uniliteref{Anhang}{#1}}
\newcommand{\fullaref}[1]{\unifullref{Anhang}{#1}}
\newcommand{\aref}[1]{\litearef{#1}}
% Abbildungen
\newcommand{\litebref}[1]{\uniliteref{Abbildung}{#1}}
\newcommand{\fullbref}[1]{\unifullref{Abbildung}{#1}}
\newcommand{\bref}[1]{\litearef{#1}}
% Code
\newcommand{\litecref}[1]{\uniliteref{Code}{#1}}
\newcommand{\fullcref}[1]{\unifullref{Code}{#1}}
\newcommand{\cref}[1]{\litearef{#1}}
% Formeln
\newcommand{\litecref}[1]{\uniliteref{Formel}{#1}}
\newcommand{\fullcref}[1]{\unifullref{Formel}{#1}}
\newcommand{\cref}[1]{\litearef{#1}}
% Kapitel
\newcommand{\fullsref}[1]{\unifullref{Kapitel}{#1}}
\newcommand{\litesref}[1]{\uniliteref{Kapitel}{#1}}
\newcommand{\sref}[1]{\litesref{#1}}
% Tabellen
\newcommand{\litetref}[1]{\uniliteref{Tabelle}{#1}}
\newcommand{\fulltref}[1]{\unifullref{Tabelle}{#1}}
\newcommand{\tref}[1]{\litearef{#1}}
% Kompatibilität
\newcommand{\literef}[1]{\litesref{#1}}
\newcommand{\fullref}[1]{\fullsref{#1}}
% Auf gleiche Art können auch eigene Kommandos in eine Datei wie 'inhalt/kommandos.tex' ausgelagert werden

%Dokument-Anfang
\begin{document}

%! #########################################
%! Inhalt der Arbeit
\frontmatter

%! automatische Auswahl der Titelseite
\ifthenelse{\isundefined{\autorzwei}}{%Titelseite
\begin{titlepage}
\begin{center}
\textbf{\Huge \begin{center} Praxisbeleg \end{center}}
\LARGE{\titel \\}
\vspace{1.0cm}
\end{center}
\begin{flushleft}
\large{
\begin{tabular}{l l r}
\vspace{0.7cm}
\textbf{Vorgelegt am:}\quad\quad\quad & \abgabedatum\\
\textbf{Von:}           ~ & \textbf{\autoreins}\\
~ & \autorstrasse \\
\vspace{0.7cm}
~ & \autorort \\
\textbf{Studiengang:}   ~ & \studiengang \\
\vspace{0.7cm}
\textbf{Studienrichtung:} ~ & \studienrichtung \\
\vspace{0.7cm}
\textbf{Seminargruppe:} ~ & \seminargruppe \\
\vspace{0.7cm}
\textbf{Matrikelnummer:} ~ & \matnumeins \\
\textbf{Praxispartner:} ~ & \institutioneins \\
~ & \partnerstrasse \\
\vspace{0.7cm}
~ & \partnerort \\
\textbf{Gutachter:}     ~ & \betreuereins \\ ~ & (\institutioneins)\\
                        ~ & \betreuerzwei \\ ~ & (\institutionzwei)\\
\vspace{0.7cm}
\end{tabular}}
\end{flushleft}
\end{titlepage}
\newpage
}{%
    \ifthenelse{\isundefined{\autordrei}}{%Titelseite

\begin{titlepage}
\begin{center}

\textbf{\Huge Projektarbeit}\\
\vspace{1.5cm}
\LARGE{\titel \\}
\vspace{1.5cm}
\end{center}
\begin{flushleft}
\large{
\begin{tabular}{l l r}
\vspace{1.0cm}
\textbf{Vorgelegt am:}\quad\quad\quad & \abgabedatum\\

\textbf{Von:}           ~ & \textbf{\autoreins}\\
\vspace{1.0cm}
                        ~ & \textbf{\autorzwei}\\

\textbf{Studiengang:}   ~ & \studiengang \\
\vspace{1.0cm}
\textbf{Studienrichtung:} ~ & \studienrichtung \\
\vspace{1.0cm}
\textbf{Seminargruppe:} ~ & \seminargruppe \\

\textbf{Matrikelnummer:} ~ & \matnumeins \\
\vspace{1.0cm}
                         ~ & \matnumzwei \\

\textbf{Gutachter:}     ~ & \betreuereins \\ ~ & (\institutioneins)\\
                        ~ & \betreuerzwei \\ ~ & (\institutionzwei)\\

\end{tabular}}
\end{flushleft}
\end{titlepage}
\newpage}{%
        \ifthenelse{\isundefined{\autorvier}}{%Titelseite

\begin{titlepage}
\begin{center}

\textbf{\Huge Projektarbeit}\\
\vspace{1.5cm}
\LARGE{\titel \\}
\vspace{1.5cm}
\end{center}
\begin{flushleft}
\large{
\begin{tabular}{l l r}
\vspace{1.0cm}
\textbf{Vorgelegt am:}\quad\quad\quad & \abgabedatum\\

\textbf{Von:}           ~ & \textbf{\autoreins}\\
                        ~ & \textbf{\autorzwei}\\
\vspace{1.0cm}
                        ~ & \textbf{\autordrei}\\

\textbf{Studiengang:}   ~ & \studiengang \\
\vspace{1.0cm}
\textbf{Studienrichtung:} ~ & \studienrichtung \\
\vspace{1.0cm}
\textbf{Seminargruppe:} ~ & \seminargruppe \\

\textbf{Matrikelnummer:} ~ & \matnumeins \\
                         ~ & \matnumzwei \\
\vspace{1.0cm}
                         ~ & \matnumdrei \\
\textbf{Gutachter:}     ~ & \betreuereins \\ ~ & (\institutioneins)\\
                        ~ & \betreuerzwei \\ ~ & (\institutionzwei)\\
                        
\end{tabular}}
\end{flushleft}
\end{titlepage}
\newpage}{\include{vorlage/Titelseite_4Autoren}}
    }
}

\ifthenelse{\isundefined{\jahr}}{}{\chapter*{Themenblatt}
\rohead{\textnormal{Themenblatt}}
\addcontentsline{toc}{chapter}{Themenblatt}
\textcolor{red}{\large{Dies ist ein Platzhalter für das Themenblatt, welches durch den Vorsitzenden des Prüfungsausschusses ausgestellt wird. Diese Seite ersetzen!}}
\normalsize
\normalcolor
\clearpage
\rohead{\textnormal{\headmark}}}

%! Nicht benötigte Verzeichnisse hier auskommentieren
%? Inhaltsverzeichnis
\vfuzz=5pt
\tableofcontents
\newpage
\vfuzz=0.1pt

%? Abbildungsverzeichnis
\listoffigures
\newpage

%? Tabellenverzeichnis
\listoftables
\newpage

%? Programmcodeverzeichnis
\listofcodes
\newpage

%? Formelverzeichnis
\listofformeln
\newpage

%? Abkürzungsverzeichnis
%Abkürzungsverzeichnis, Abkürzungen hier entsprechend des Beispiels eintragen und dann im Fließtext mit
% \ac{Kürzel} nutzen
\addsec{Abkürzungsverzeichnis}
\begin{acronym}
\acro{SSH}{Secure Shell}
\end{acronym}

%! Hier beginnt der eigentliche Inhalt der Arbeit.
\mainmatter

%! Eine der folgenden Zeilen wieder aktivieren und entsprechende Datei ablegen
%? druckt Firmenlogo ab Kapitel 1 in Kopfzeile - SVG ist, auch bei kleinen Grafiken, zu bevorzugen, wenn vorhanden
%\lohead{\includesvg[height=8mm,inkscapelatex=false]{bilder/firmenlogo}}
%\lohead{\includegraphics[height=8mm]{bilder/firmenlogo}}

% Es ist möglich, die ganze Arbeit in eine Datei (z.B. "Inhalt.tex") zu schreiben,
% allerdings empfiehlt es sich, zur besseren Strukturierung mehrere Dateien, bspw. eine pro Kapitel, zu verwenden.
% Diese werden folgendermaßen eingebunden:
%\include{inhalt/Inhalt}

%! Doku-/Testinhalt, diese Zeile bei Nutzung der Vorlage entfernen
%Alle Kapitel beginnen mit der Hauptüberschrift
\chapter{Einleitung}
Diese PDF wurde mit der Vorlage erstellt, um die Funktion und Formatierung dieser zu zeigen.

Die Vorlage\onlinezitat{Vorlage} ist ein Gemeinschaftsprojekt im Rahmen unseres Studiums.
Der Docker-Container\onlinezitat{HILLE2021} gehört dazu.
Die Vorlage richtet sich weitestgehend nach dem Dokument \ac{HAWA}\onlinezitat{HAWA} der \href{https://www.ba-glauchau.de/}{Staatlichen Studienakademie Glauchau}.

Weitere Hinweise befinden sich in der README.md oder im \href{https://github.com/DSczyrba/Vorlage-Latex/wiki}{Wiki}.
Eine ausführliche Dokumentation zu diesem Dokument wird folgen.

\chapter{Beispiele}
\label{sec:beispiele}
\section{Pixelgrafiken}
Siehe Quelltext.\fn{Dort wird das Kommando \striche{bild} aufgeführt. In dieser Vorlage sind allerdings keine Bilder (im vorgesehenen Ordner) enthalten.}
%\bild[0.5]{dateiname-ohne-endung}{Text zu einem Bild}{label1}
\section{Vektorgrafiken}
Siehe Quelltext.
%\svg[0.5]{dateiname-ohne-endung}{Text zu einer \ac{SVG}-Datei}{label2}
\section{Programmcode}
\begin{code}
    \begin{minted}{python}
        import asdf
        from foo import bar
        
        def yolo():
            return "glhf"
        
        class Wooo(Foo):
            def __init__(self, boo):
                self.doo = boo
                moo = 1 + 2 +3
    \end{minted}
    \caption[Beispielcode]{Beispielcode}
    \label{code:example}
\end{code}

An dieser Stelle wird noch ein einzeiliger Code eingefügt, der keine Zeilennummerierung erhalten soll.
Dies kann genutzt werden, wenn man nur kurz auf eine Funktion eingehen möchte und diese nicht im Quellcodeverzeichnis erscheinen soll.
\mint{python}|print("Hallo Vorlage")|

Ein weiterer Code um zu schauen, ob danach die Zeilennummerierung wieder funktioniert.

\begin{code}
  \begin{minted}{python}
      class Wooo(Foo):
          def __init__(self, boo):
              self.doo = boo
              moo = 1 + 2 +3
  \end{minted}
  \linkcaption{Beispielcode}
  \label{code:example2}
\end{code}
\vgcaption{https://link-wo-es-den-code-gibt.de}{06.07.2022}
\section{Ordnerstruktur}
    In diesem Abschnitt wird eine Ordnerstruktur in \literef{beispielbaum} gezeigt.
    Ordnerstrukturen können im Quellcode definiert werden.
    Die Symbole für Ordner und Dateien können, auf Wunsch, ausgetauscht oder erweitert werden, um verschiedene Dateitypen abzubilden.
    \verzeichnis{%
          .1 \dtfolder Vorlage-Latex. .2 \dtfile HINWEISE.md.
          .2 \dtfile LICENSE.
          .2 \dtfile README.md.
          .2 \dtfile sortieren.py.
          .2 \dtfolder Latex.
            .3 \dtfolder bilder.
              .4 \dtfile firmenlogo.svg.
            .3 \dtfolder inhalt.
              .4 \dtfile Abkürzungen.tex.
              .4 \dtfile Anhang.tex.
            .3 \dtfolder light.
              .4 \dtfile main.tex.
              .4 \dtfile README.md.
              .4 \dtfile vorlage\_light.tex.
            .3 \dtfile literatur.bib.
            .3 \dtfile main.tex.
            .3 \dtfile main\_abstract.tex.
            .3 \dtfile metadaten.sty.
    }{Ein Verzeichnis-Baum}{beispielbaum}

    \section{Formeln}
    Formeln werden mittels \striche{\textbackslash formula} in die Arbeit eingebunden.
    Die Beschriftungen befinden sich, wie es die HAWA\onlinezitat[Abs. 3.3.3.7]{HAWA} verlangt, rechts neben der Formel.
    \formula{$\Delta p_{WZ}=\Delta p\cdot\dfrac{\dot{V}^2_S}{\dot{V}^2_G}$}{%
    $\dot{V}^2_S = $ Spitzendurchfluss $\left[ m^3/h\right]$\\
    $\dot{V}^2_G = $ maximaler Durchfluss im Wasserzähler $\left[ m^3/h\right]$\\
    $\Delta p = $ Druckverlust bei $V_{max} \left[bar\right]$}{Druckverlust}{formel:ohm}

    \section{Überschriften}
    \subsection{in verschiedenen}
    \subsubsection{Größen}
    Kapitelüberschriften werden per \emph{\textbackslash chapter}, Abschnitte per \emph{\textbackslash section}, Unterabschnitte per \emph{\textbackslash subsection} und Unterunterabschnitte per \emph{\textbackslash subsubsection} eingefügt.

\chapter{Test-/Dokutabelle}
Diese Tabelle dient hauptsächlich zum Testen der einzelnen Kommandos, sowie als minimales Beispiel für eine \emph{tabularx}-Tabelle:
\hbadness=10000 %"Tabelle 1" in Spalte Beispiel erzeugt sonst eine Warnung
\begin{table}
\begin{tabularx}{\columnwidth}{|p{3cm}|X|p{.2\columnwidth}|}
\hline
Gegenstand & Beispiel & Befehl \\
\hline
Kurzer Verweis & \literef{sec:beispiele} & \emph{\textbackslash literef}\\
\hline
Langer Verweis & \fullref{beispielbaum} & \emph{\textbackslash fullref}\\
\hline
Verweis ohne Objektname & \autoref{beispieltabelle} & \emph{\textbackslash autoref}\\
\hline
\multicolumn{2}{|c|}{verbundene Spalten mit zentriertem Text} & \emph{\textbackslash multicolumn} \\
\hline
Zeile 1 & \multirow{2}{\hsize}{verbundene Zeilen inklusive automatischer Zeilenumbrüche} & \emph{\textbackslash multirow} \\
\cline{1-1}\cline{3-3}
Zeile 2 & & mit \emph{\textbackslash hsize}\\
\hline
\end{tabularx}
\caption{Beispieltabelle}
\label{beispieltabelle}
\end{table}
\hbadness=1000

In diesem Satz wird eine Quelle mit nur einem Autor zitiert\onlinezitat{HAWA}, eine Quelle mit drei Autoren\vgonlinezitat{Vorlage} wird sinngemäß zitiert und eine Quelle mit vier Autoren\onlinezitat[Abs.~2.3]{rfc3596}, welche als \emph{Autor 1; u.a.} dargestellt werden sollte, existiert ebenfalls. Es folgen ein Buch-Zitat mit Seitenangabe\zitat[9]{KOHM2020} und ein sinngemäßes Zitat aus einer unveröffentlichten Quelle\vguvzitat{UV}.


%? Quellen- und Literaturverzeichnis
% Quellen neigen dazu, zu breite Zeilen durch (z.B.) schlecht umgebrochene URLs zu erzeugen, die sich kaum vermeiden lassen. --> Ignorieren
\hfuzz=10pt
\printbibliography
\hfuzz=0.1pt

%? Anhang
%!	Anhang

\clearpage
\appendix
\clearpage

%! Section Befehl wird umgeschrieben, damit keine Überschriften mehr angezeigt werden
%!Kann falls Überschriften gewollt sind entfern werden oder erst später eingefügt
% Beginn 
\renewcommand{\section}[1]{
\par\refstepcounter{section}
\sectionmark{#1}
\addcontentsline{atoc}{section}{\protect\numberline{\thesection}#1}
\lohead{\textnormal{#1}}
} % Ende

%! Anhang 1
\section{Erster toller Anhang}
Hihi hier kommt eigentlich ein Anhangverzeichnis hin :D
\newpage

%! Anhang 2
\section{Inhalt der CD}
CD mit folgenden Inhalten:
\begin{itemize}
	\item dieses Dokument
	\item Latex Dateien
	\item Youtube-Video als Bonus
\end{itemize}
 \newpage

%! Eidestattliche Erklärung

%	Eidesstattliche Erklärung

\cleardoublepage 
\section{Ehrenwörtliche Erklärung}
\begin{bfseries}
	\begin{center}
		%! Hier muss noch Abstand rein!
		\Huge{Ehrenwörtliche Erklärung}\\[3cm]
	\end{center}
\end{bfseries}
			\begin{tabbing}
		Ich erkläre \= ehrenwörtlich,\\[1cm]
		1. 	\> dass ich meinen Praxisbeleg mit dem Thema:\\[1cm]
		   	\> Thema hier\\[1cm]
		ohne fremde Hilfe angefertigt habe;\\[1cm]
		
		2.	\> dass ich die Übernahme wörtlicher Zitate aus der Literatur sowie die\\ 		  
			\>Verwendung der Gedanken anderer Autoren an den entsprechenden\\
			\> Stellen innerhalb der Arbeit gekennzeichnet habe;\\[0.5cm]
		
		3.	\> dass ich meine Praxisbeleg bei keiner anderen Prüfung vorgelegt habe. \\[1cm]
		Ich bin mir bewusst, dass eine falsche Erklärung rechtliche Folgen haben wird. \\[2cm]
		\end{tabbing}
		
		 \begin{tabular}{p{8cm}l}
		  ----------------------------------- &  ----------------------------------- \\
		  Ort, Datum & Unterschrift  \\
		 \end{tabular}

%!##########################################

\end{document}
