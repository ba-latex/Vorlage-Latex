% main.tex - Hauptdatei der Vorlage

% An diese Stelle muss ein Schalter
%! LaTeX Vorlage
\documentclass[12pt, fleqn, captions=nooneline, titlepage, footsepline, headsepline, toc=chapterentrywithdots, listof=entryprefix, bibliography=totoc, parskip=half-]{scrreprt}

\usepackage{silence} %unnötige Warnungen unterdrücken
\WarningFilter{latex}{You have requested}
% \WarningFilter{scrlayer-scrpage}{\headheight to low}
% \WarningFilter{scrlayer-scrpage}{\footheight to low}
% \WarningFilter{scrlayer-scrpage}{Very small head height detected}
% \WarningFilter{fvextra}{} % was caused by loading csquotes before minted (which loads fvextra)
% \WarningFilter{lineno}{}

\pdfsuppresswarningpagegroup=1

\ProvidesPackage{metadaten}
\usepackage{metadaten}

\usepackage{tocbasic}
\usepackage[ngerman]{babel}
\usepackage[%
  backend=biber,
  labeldateparts=true,
  style=authortitle,
  isbn=false,
  dashed=false,
  maxnames=3]{biblatex}

%!  Änderungen für scrreprt
\renewcommand{\autodot}{}
\usepackage{chngcntr}
\counterwithout{figure}{chapter}
\counterwithout{table}{chapter}
\counterwithout{footnote}{chapter}
\RedeclareSectionCommand[style=section,afterskip=.15em]{chapter}
\setcounter{secnumdepth}{\subsubsectionnumdepth}
\setcounter{tocdepth}{\subsubsectionnumdepth}
\addtokomafont{chapter}{\LARGE}
\addtokomafont{section}{\Large}
\addtokomafont{subsection}{\large}
\addtokomafont{subsubsection}{\normalsize}
\renewcommand*{\chaptermarkformat}{}

\input{vorlage/vorlage_subs/verzeichnisse}

%! Ermöglicht die Ausgabe des aktuellen Titels.
\usepackage{nameref}
\makeatletter
\newcommand*{\currentname}{\@currentlabelname}
\makeatother

%! zusätzliche LaTeX-Packages
\usepackage[table]{xcolor}
\usepackage{amsmath}
\usepackage{amssymb}
\usepackage{amsthm}
\usepackage{tabularx}
\usepackage{multirow}
\usepackage{booktabs}
\usepackage{svg}
\usepackage{graphicx}
\usepackage{float}
%! floatplacement für Codes muss in minted.tex definiert werden
\floatplacement{figure}{htbp}
\floatplacement{table}{htbp}
\floatplacement{figure}{htbp}
\floatplacement{formel}{htbp}
\usepackage[a4paper,lmargin={2.5cm},rmargin={2.5cm},tmargin={2cm},bmargin={2cm}]{geometry}
\usepackage{lineno}
\usepackage[T1]{fontenc}
\usepackage{listings}
\usepackage{tikz}
\usepackage{varwidth}
\usepackage{ifthen}
\usepackage{etoolbox}

%! Abbildungen von Verzeichnisstrukturen
\input{vorlage/vorlage_subs/dirtree}

%! PageBreaks nach jeder Section
\let\oldchapter\chapter
\renewcommand\chapter{\clearpage\oldchapter}

%! Code-Umgebungen
%! Code-Integration im Dokument
%? Inklusive Erzeugung eines Custom-Enviroments für Programmcodes
\usepackage{minted}

% ! Anpassung der Minted-Umgebung, um Abstände zu vereinheitlichen und die Optik zu verbessern
\let\oldminted\minted
\let\oldendminted\endminted
\def\minted{\begingroup \vspace{-0.3cm} \oldminted}
\def\endminted{\oldendminted \vspace{-0.5cm} \endgroup}

\xapptocmd{\inputminted}{\vspace{-0.5cm}}{}{}

%Zeilennummern neu definieren, um Warnungen zu vermeiden und modern anmutende Zahlen zu nutzen
\renewcommand{\theFancyVerbLine}{\scriptsize{\arabic{FancyVerbLine}}}

%Standardformatierung für minted-Umgebung erstellen
\setminted{
    tabsize=2,
    breaklines,
    autogobble,
    fontfamily=courier,
    linenos,
    %! see https://pygments.org/demo/#try, other nice options: paraiso-light, solarized-light, rainbow_dash, gruvbox-light, stata, tango
    style=emacs,
    fontsize=\footnotesize
}
%Keine Zeilennnummer, wenn der einzeilige \mint-Befehl genutzt wird
\xpretocmd{\mint}{\setminted{linenos=false}}{}{}
\xpretocmd{\minted}{\setminted{linenos=true}}{}{}

\DeclareNewTOC[
  type=code,                           % Name der Umgebung
  types=codes,                         % Erweiterung (\listofschemes)
  float,                               % soll gleiten
  floatpos=htbp,
  tocentryentrynumberformat=\bfseries, % voreingestellte Gleitparameter
  name=Code,                           % Name in Überschriften
  listname={Programmcodeverzeichnis},  % Listenname
  % counterwithin=chapter
]{loc}
\setuptoc{loc}{totoc}

\renewcommand*{\codeformat}{%
  \codename~\thecode%
  \autodot{\space\space\space}
}
\BeforeStartingTOC[loc]{\renewcommand*\autodot{\space\space\space\space}}
\AtBeginEnvironment{minted}{\vspace{\baselineskip}}


\usepackage{scrhack} %necessary to use float with scrreprt
\usepackage{csquotes}

%! Schriftart
%! Die HAWA schreibt Arial vor, welches in Standard LaTeX-Distributionen nicht mitgeliefert wird. Helvetica ist nahezu identisch.
\usepackage{helvet}
\usepackage{microtype}
\renewcommand{\familydefault}{\sfdefault}
%! Hyperref und PDF-meta
\usepackage[hidelinks]{hyperref}

%! PDF-Metadaten (Autor/Titel/Beschreibung)
\input{vorlage/vorlage_subs/pdf_metadaten}
\usepackage[numbered]{bookmark}
\usepackage[printonlyused]{acronym}
\usepackage{enumitem}

%Abkürzungsverzeichnis - Formatierung (bspw. zur korrekten Anzeige von BASH-Befehlen)
\renewcommand*{\aclabelfont}[1]{\acsfont{#1}}

%! Untertitel/Legenden
\input{vorlage/vorlage_subs/captions}

%! Literaturverzeichnis und Zitierbefehle
\input{vorlage/vorlage_subs/literatur_zitate}

%! Kopf- und Fußzeilen
\input{vorlage/vorlage_subs/kopf_fußzeile}

%! verbesserte Umbrüche (hoffentlich)
\input{vorlage/vorlage_subs/umbrüche}

%? wird am Ende geladen, um einheitliche Spacings sicherzustellen
\usepackage{setspace}
%! setzt den Zeilenabstand in Floats (Tabellen, Code-Listings usw.) auf den in LaTeX definierten Wert (statt 1.0)
\makeatletter
\let\@xfloat=\latex@xfloat
\makeatother


%Dokument-Anfang
\begin{document}

%! #########################################
%! Inhalt der Arbeit
\frontmatter

%! Auswahl der Titelseite
\include{inhalt/Titelseite_Praxisbeleg}
\include{inhalt/Titelseite_3Autoren}

%! Nicht benötigte Verzeichnisse hier auskommentieren
%? Inhaltsverzeichnis
\vfuzz=5pt
\tableofcontents
\newpage
\vfuzz=0.1pt

%? Abbildungsverzeichnis
\listoffigures
\newpage

%? Tabellenverzeichnis
\listoftables
\newpage

%? Programmcodeverzeichnis
\listofcodes
\newpage

%? Formelverzeichnis
\listofformeln
\newpage

%? Abkürzungsverzeichnis
\include{inhalt/Abkürzungen}

%! Hier beginnt der eigentliche Inhalt der Arbeit.
\mainmatter

%! Eine der folgenden Zeilen wieder aktivieren und entsprechende Datei ablegen
%? druckt Firmenlogo ab Kapitel 1 in Kopfzeile - SVG ist, auch bei kleinen Grafiken, zu bevorzugen, wenn vorhanden
%\lohead{\includesvg[height=8mm,inkscapelatex=false]{bilder/firmenlogo.svg}}
%\lohead{\includegraphics[height=8mm]{bilder/firmenlogo.png}}

%! Es ist möglich, die ganze Arbeit in eine Datei ("Kapitel1.tex") zu schreiben.
%! Sollten mehr Struktur und Übersicht gewünscht werden, können noch zusätzliche Dateien angelegt werden,
%! diese müssen nach dem selben Schema wie "Kapitel1.tex" hinzugefügt werden:
\usemintedstyle{colorful} %Festlegen des Minted Styles

\section{Einleitung}
  Diese PDF wurde mit der Vorlage erstellt, um die Funktion und Formatierung dieser zu zeigen.

  Die Vorlage\onlinezitat{SCZYRBA2020} ist ein Gemeinschaftsprojekt im Rahmen unseres Studiums.
  Der Docker-Container\onlinezitat{HILLE2021} gehört dazu.
  Die Vorlage richtet sich weitestgehend nach dem Dokument \ac{HAWA}\onlinezitat{HAWA} der \href{https://www.ba-glauchau.de/}{Staatlichen Studienakademie Glauchau}.

  Weitere Hinweise befinden sich in der README.md oder im \href{https://github.com/DSczyrba/Vorlage-Latex/wiki}{Wiki}.
  Eine ausführliche Dokumentation zu diesem Dokument wird folgen.

\section{Dokumentation der Vorlage}
  \subsection{Allgemeiner Aufbau}
    \subsubsection{Ordnerstruktur}
    \subsubsection{vorlage.tex}
    \subsubsection{Kapitel.tex}
    \subsubsection{Grundsätzliche Hinweise}
  \subsection{Einbinden von Grafiken}
    \subsubsection{Pixelgrafiken}
      Es ist sowohl möglich Pixelgrafiken wie in diesem Kapitel beschrieben 
      oder auch wie im nächsten Kapitel \ref{sec:vectorgrafiken} beschrieben Vektorgrafiken einzubinden.

      Prinzipiell gibt es zwei Möglichkeiten eine Grafik einzubinden.
      Zum einen über eine vordefinierte Funktion namens \mintinline{latex}{\bild{}}. 
      Diese benutzt im Endeffekt die zweite Variante ist durch den kürzeren Befehl allerdings deutlich einfacher zu nutzen.
            
      \begin{code}[H]
        \begin{minted}[linenos, breaklines, frame=none, numbers=left, autogobble, numbersep=5pt]{latex}
          \bild[0.5]{ba-gc-logo}{Text zu einem Bild}{fig:ba-gc-logo}
          % \bild ruft die Funktion auf
          % 0.5 steht für die Breite des Bildes (0.0 bis 1.0)
          % ba-gc-logo gibt den Dateinamen des Bildes im Ordner bilder an
          % label beschreibt den Namen für spätere \refs{}
        \end{minted}
        \caption{Grafiken mittels Bild-Funktion einbinden}
        \label{code:bild-einfuegen1}
        \end{code}

        Zum anderen über die \LaTeX-Standardvariante einer Gleitumgebung, im Folgenden \striche{Environment} genannt.

        \begin{code}[H]
          \begin{minted}[linenos, breaklines, frame=none, numbers=left, autogobble, numbersep=5pt]{latex}
            \begin{figure}[H]
              \centering 
              \includegraphics[width=0.5\columnwidth]{bilder/ba-gc-logo}
              \caption{Bildunterschrift}
              \label{fig:ba-gc-logo}
            \end{figure}
          \end{minted}
          \caption{Grafiken mittels Environments einfügen}
          \label{code:bild-einfuegen2}
        \end{code}

        In der folgenden Abbildung wird die in den Codes beschriebene Grafik eingebunden.
        \bild[0.5]{ba-gc-logo}{Text zu einem Bild}{label1}

  \subsubsection{Vektorgrafiken}
    \label{sec:vectorgrafiken}
    Vektorgrafiken verhalten sich prinzipiell ähnlich wie im vorherigen Kapitel beschriebene Pixelgrafiken.
    Dennoch sollen hier aus Gründen der Vollständigkeit beide Varianten gezeigt werden.

    \begin{code}[H]
      \begin{minted}[linenos, breaklines, frame=none, numbers=left, autogobble, numbersep=5pt]{latex}
        \svg[0.5]{ba_glauchau_logo}{Text zu einer SVG-Datei}{fig:ba-gc-logo}
        % \svg ruft die Funktion auf
        % 0.5 steht für die Breite des Bildes (0.0 bis 1.0)
        % ba_glauchau_logo gibt den Dateinamen des Bildes im Ordner bilder an
        % label beschreibt den Namen für spätere \refs{}
      \end{minted}
      \caption{SVG-Grafiken mittels Bild-Funktion einbinden}
      \label{code:svg-einfuegen1}
      \end{code}

      \begin{code}[H]
        \begin{minted}[linenos, breaklines, frame=none, numbers=left, autogobble, numbersep=5pt]{latex}
          \begin{figure}[H]
            \centering
            \includesvg[width=0.5\columnwidth,inkscapelatex=false]{bilder/ba}
            \caption{Bildunterschrift}
            \label{fig:ba-gc-logo}
          \end{figure}
        \end{minted}
        \caption{SVG-Grafiken mittels Environments einfügen}
        \label{code:bildsvg-einfuegen2}
      \end{code}

      In der folgenden Abbildung wird die in den Codes beschriebene SVG-Grafik eingebunden.
      \svg[0.5]{ba_glauchau_logo}{Text zu einer \ac{SVG}-Datei}{label2}

\section{Kapitel 1}
  \blindtext


%? Quellen- und Literaturverzeichnis
% Quellen neigen dazu, zu breite Zeilen durch (z.B.) schlecht umgebrochene URLs zu erzeugen, die sich kaum vermeiden lassen. --> Ignorieren
\hfuzz=10pt
\printbibliography
\hfuzz=0.1pt

%? Anhang
%! Anhang

\clearpage
\appendix
\clearpage

%! Chapter Befehl wird umgeschrieben, um Überschriften zu verbergen
%! Kann, falls Überschriften gewollt sind, entfernt oder erst später eingefügt werden.
% Beginn
\makeatletter
\renewcommand{\chapter}[1]{%
\par\refstepcounter{chapter}%
\sectionmark{#1}%
\NR@gettitle{#1}%<---------
\addcontentsline{atoc}{chapter}{\bfseries\protect\numberline{\thechapter}{\mdseries#1}}%
\lohead{\textnormal{#1}}%
}
\makeatother
% Ende

%! Anpassung der Darstellung von Abbildungen im Anhang
%! Eine Variante auskommentieren
%? Möglichkeit 1: ohne Nummerierung
%\renewcommand{\bild}[4][1.0]{\begin{figure}[H]
    %\centering
    %\includegraphics[width=#1\columnwidth]{bilder/#2}
    %\caption*{\bfseries Abbildung \mdseries #3}
    %\label{#4}
    %\end{figure}}

%? Möglichkeit 2: mit Nummerierung aber nicht im Abbildungsverzeichnis
\renewcommand{\bild}[4][1.0]{\begin{figure}[H]
    \centering
    \includegraphics[width=#1\columnwidth]{bilder/#2}
    \caption[]{#3}
    \label{#4}
    \end{figure}}

%! CD/USB-Inhalt, für USB-Stick entsprechend anpassen, Icon ein-/ CD ausblenden
\chapter{Inhalt der CD}
\label{cd-inhalt}
\renewcommand\DTstyle{\sffamily}
\renewcommand{\DTstyle}{\textrm\expandafter\raisebox{-0.7ex}}
\dirtree{%
.1 
%! CD
\begin{tikzpicture}
    \draw circle (0.16);
    \draw circle (0.07);
    \draw circle (0.02);
\end{tikzpicture}
%! USB
%\hspace{.2em}\dtusb
\raisebox{.05em}{ CD mit folgenden Inhalten:}.
.2 \dtfolder Anhänge.
.2 \dtfolder LaTeX-Quellen.
.2 \dtfolder Online-Quellen.
.2 \dtfile dieses Dokument.
.2 \dtfile \href{https://www.youtube.com/watch?v=dQw4w9WgXcQ}{YouTube-Video} als Bonus.
}
\vspace*{\fill}
\begin{center}
    \begin{tikzpicture}
        \draw (0,0) rectangle (12.2,12.2);
        \draw (6.1,6.1) circle (5.5);
        \draw (6.1,6.1) circle (.75);
        \draw (6.1,6.1) circle (2.4);
    \end{tikzpicture}
\end{center}
\vspace*{\fill}
\clearpage

% Warnungen für vorgefertigte Dokumente deaktivieren
\hbadness=10000
% Eidesstattliche Erklärung, Erklärung Plagiatsprüfung
\input{vorlage/Erklärung}

% Warnungen zurücksetzen
\hbadness=1000
%!##########################################

\end{document}
