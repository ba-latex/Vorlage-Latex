% main.tex - Hauptdatei der Vorlage

% An diese Stelle muss ein Schalter


\documentclass[12pt, captions=nooneline, titlepage, footsepline, headsepline, toc=sectionentrywithdots, listof=entryprefix, bibliography=totoc]{scrartcl}
\usepackage{tocbasic}
\usepackage[ngerman]{babel}
\usepackage[backend=biber, style=authortitle]{biblatex}

%! Das Inhaltsverzeichnis wird an dieser Stelle formatiert.
\RedeclareSectionCommands[tocindent=0pt]{section, subsection, subsubsection}
\RedeclareSectionCommands[tocnumwidth=45pt]{section, subsection, subsubsection}


%! Formatierung aller Verzeichnisse
\renewcaptionname{ngerman}{\refname}{Quellenverzeichnis}
\setuptoc{toc}{totoc}
\setuptoc{lof}{totoc}
\setuptoc{lot}{totoc}
%\setuptoc{bibliography}{totoc}
\renewcommand*\listoflofentryname{\bfseries\figurename}
\BeforeStartingTOC[lof]{\renewcommand*\autodot{\space\space\space\space}}
\addtokomafont{captionlabel}{\bfseries}
\renewcommand*\listoflotentryname{\bfseries\tablename}
\BeforeStartingTOC[lot]{\renewcommand*\autodot{\space\space\space\space}}

%! Latex-Packages, die verwendet werden

\usepackage{amsmath}
\usepackage{amssymb}
\usepackage{amsthm}
\usepackage{tabularx}
\usepackage{setspace} 
\usepackage{booktabs}
\usepackage{svg}
\usepackage{graphicx}
\usepackage{float}
\usepackage[a4paper,lmargin={2.5cm},rmargin={2.5cm},tmargin={2cm},bmargin={2cm}]{geometry}

\usepackage{csquotes}


%! Schriftart
\usepackage{helvet}
\usepackage{microtype}
\renewcommand{\familydefault}{\sfdefault}

\usepackage[hidelinks]{hyperref}
\usepackage{acronym}
\renewcommand*{\aclabelfont}[1]{\acsfont{#1}} %Abkürzungsverzeichnis - Formatierung
\usepackage{listings} %zB zum korrekten anzeigen von BASH-Befehlen
\lstset{ %listings Einstellungen
  breaklines=true,
  postbreak=\mbox{\textcolor{red}{$\hookrightarrow$}\space},
}



%! Caption um Tabellen und Abbildungen richtig zu beschriften
\usepackage{caption}
\captionsetup{labelsep=none}
%\addto\captionsngerman{\renewcommand{\figurename}{Abbildung}}
\renewcommand*{\figureformat}{%
  \figurename~\thefigure%
  \autodot{\space\space\space}
}
\renewcommand*{\tableformat}{%
  \tablename~\thetable%
  \autodot{\space\space\space}
}

%! Formatierung des Literaturverzeichnis
\DeclareFieldFormat{url}{\newline\url{#1}}
\DeclareFieldFormat{urldate}{\addcomma\space\bibstring{urlseen}\space#1}

\DefineBibliographyStrings{german}{%
  urlseen = {Abruf am},
}
\setlength\bibitemsep{\baselineskip}

%! Formatierung der Fußnotenzitate / Literaturverzeichnis
\renewcommand*{\newunitpunct}{\addcomma\space} 

%? Normales Zitat...
\DeclareCiteCommand{\zitat}[\mkbibfootnote]
  {\usebibmacro{prenote}}
  {\usebibmacro{citeindex}%
   %\mkbibbrackets{\usebibmacro{cite}}%
   \setunit{\addnbspace}
   \printnames{labelname}%
   \setunit{\labelnamepunct}
   %\printfield[citetitle]{title}%
   \newunit
   \printfield{year}
   \newunit
   \printfield{pages}}
  {\addsemicolon\space}
  {\usebibmacro{postnote}}

%? Online Zitat
\DeclareCiteCommand{\onlinezitat}[\mkbibfootnote]
  {\usebibmacro{prenote}}
  {\usebibmacro{citeindex}%
   %\mkbibbrackets{\usebibmacro{cite}}%
   \setunit{\addnbspace}
   online:
   \printnames{labelname}%
   \setunit{\labelnamepunct}
   %\printfield[citetitle]{title}%
   \newunit
   \printfield{year}
   \printtext{(}\printfield{urlday}.\printfield{urlmonth}\printtext{.}\printfield{urlyear}\printtext{)}}
   %\printurldate}
  %{\addsemicolon\space}
  {\usebibmacro{postnote}}


\DeclareMultiCiteCommand{\zitate}[\mkbibfootnote]{\footpartcite}{\addsemicolon\space}

\addbibresource{literatur.bib}

%! Kopf- und Fußzeile
\usepackage[automark]{scrlayer-scrpage} 
\pagestyle{scrheadings} 
\clearscrheadings 
\clearscrplain 
\rohead{\headmark} 
\lofoot{} 
\cofoot{} 
\rofoot{\pagemark}
\makeatletter
\usepackage{geometry}
\geometry{a4paper,
          left=25mm,right=25mm,top=20mm,bottom=20mm,
          includehead=false, % Kopfzeile außerhalb des Textkörper, also im Rand
          includefoot=false,
          headheight = \baselineskip,
          headsep = \dimexpr\Gm@tmargin-\headheight-10mm,
          footskip = \dimexpr\Gm@bmargin-10mm,
          %showframe,
          bindingoffset=0mm}
% Kopfzeile 1,0 cm Abstand zum Blattrand
% Fußzeile 1,0 cm Abstand zum Blattrand
\makeatother


%! Versuch von besseren Seitenumbrüchen
\clubpenalty = 10000
\widowpenalty = 10000
\displaywidowpenalty = 10000
\widowpenalties= 3 10000 10000 150

\linespread{1.3}
\newcommand\frontmatter{%
    \cleardoublepage
  %\@mainmatterfalse
  \pagenumbering{Roman}}

\newcommand\mainmatter{%
    \cleardoublepage
 % \@mainmattertrue
  \pagenumbering{arabic}}

\newcommand\backmatter{%
  \if@openright
    \cleardoublepage
  \else
    \clearpage
  \fi
 % \@mainmatterfalse
   }

% ? Ab hier beginnen eigene Befehle um den Umgang zu erleichtern   

\newcommand{\absatz}{\vspace{12pt}\noindent}
\newcommand{\logisch}[1]{$``#1``$}
\newcommand{\bild}[3][1.0]{\begin{figure}[H]
                      \centering
                      \includegraphics[width=#1\columnwidth]{bilder/#2}
                      \caption{#3}
                      \label{fig:#3}
                      \end{figure}}

%Dokument-Anfang
\begin{document}

%! #########################################
%! Inhalt der Arbeit
\frontmatter

%! Auswahl der Titelseite
%Titelseite
\begin{titlepage}
\begin{center}
\textbf{\Huge \begin{center} Praxisbeleg \end{center}}
\LARGE{\titel \\}
\vspace{1.0cm}
\end{center}
\begin{flushleft}
\large{
\begin{tabular}{l l r}
\vspace{0.7cm}
\textbf{Vorgelegt am:}\quad\quad\quad & \abgabedatum\\
\textbf{Von:}           ~ & \textbf{\autoreins}\\
~ & Am Ende 128 \\
\vspace{0.7cm}
~ & 08371 Glauchau \\
\textbf{Studiengang:}   ~ & \studiengang \\
\vspace{0.7cm}
\textbf{Studienrichtung:} ~ & \studienrichtung \\
\vspace{0.7cm}
\textbf{Seminargruppe:} ~ & \seminargruppe \\
\vspace{0.7cm}
\textbf{Matrikelnummer:} ~ & \matnumeins \\
\textbf{Praxispartner:} ~ & \institutioneins \\
~ & Industriestraße 666 \\
\vspace{0.7cm}
~ & 5121 Fucking \\
\textbf{Gutachter:}     ~ & \betreuereins \\ ~ & (\institutioneins)\\
                        ~ & \betreuerzwei \\ ~ & (\institutionzwei)\\
\vspace{0.7cm}
\end{tabular}}
\end{flushleft}
\end{titlepage}
\newpage

%Titelseite

\begin{titlepage}
\begin{center}

\textbf{\Huge Projektarbeit}\\
\vspace{1.5cm}
\LARGE{\titel \\}
\vspace{1.5cm}
\end{center}
\begin{flushleft}
\large{
\begin{tabular}{l l r}
\vspace{1.0cm}
\textbf{Vorgelegt am:}\quad\quad\quad & \abgabedatum\\

\textbf{Von:}           ~ & \textbf{\autoreins}\\
                        ~ & \textbf{\autorzwei}\\
\vspace{1.0cm}
                        ~ & \textbf{\autordrei}\\

\textbf{Studiengang:}   ~ & \studiengang \\
\vspace{1.0cm}
\textbf{Studienrichtung:} ~ & \studienrichtung \\
\vspace{1.0cm}
\textbf{Seminargruppe:} ~ & \seminargruppe \\

\textbf{Matrikelnummer:} ~ & \matnumeins \\
                         ~ & \matnumzwei \\
\vspace{1.0cm}
                         ~ & \matnumdrei \\
\textbf{Gutachter:}     ~ & \betreuereins \\ ~ & (\institutioneins)\\
                        ~ & \betreuerzwei \\ ~ & (\institutionzwei)\\
                        
\end{tabular}}
\end{flushleft}
\end{titlepage}
\newpage

%! Nicht benötigte Verzeichnisse hier auskommentieren
%? Inhaltsverzeichnis
\vfuzz=5pt
\tableofcontents
\newpage
\vfuzz=0.1pt

%? Abbildungsverzeichnis
\listoffigures
\newpage

%? Tabellenverzeichnis
\listoftables
\newpage

%? Programmcodeverzeichnis
\listofcodes
\newpage

%? Formelverzeichnis
\listofformeln
\newpage

%? Abkürzungsverzeichnis
%Abkürzungsverzeichnis, Abkürzungen hier entsprechend des Beispiels eintragen und dann im Fließtext mit
% \ac{Kürzel} nutzen
\addsec{Abkürzungsverzeichnis}
\begin{acronym}
\acro{SSH}{Secure Shell}
\end{acronym}

%! Hier beginnt der eigentliche Inhalt der Arbeit.
\mainmatter

%! Eine der folgenden Zeilen wieder aktivieren und entsprechende Datei ablegen
%? druckt Firmenlogo ab Kapitel 1 in Kopfzeile - SVG ist, auch bei kleinen Grafiken, zu bevorzugen, wenn vorhanden
%\lohead{\includesvg[height=8mm,inkscapelatex=false]{bilder/firmenlogo.svg}}
%\lohead{\includegraphics[height=8mm]{bilder/firmenlogo.png}}

%! Es ist möglich, die ganze Arbeit in eine Datei ("Kapitel1.tex") zu schreiben.
%! Sollten mehr Struktur und Übersicht gewünscht werden, können noch zusätzliche Dateien angelegt werden,
%! diese müssen nach dem selben Schema wie "Kapitel1.tex" hinzugefügt werden:
%Alle Seiten beginnen mit der Oberüberschrift
\section{Einleitung}
Die ist eine PDF mit der Vorlage erstellt, um die Funktion und Formatierung dieser zu zeigen.

Diese Vorlage\onlinezitat{SCZYRBA2020} ist ein Gemeinschaftsprojekt im Rahmen unseres Studiums.
Der Docker-Container\onlinezitat{HILLE2021} gehört dazu.

Weitere Hinweise befinden sich in der Readme.md oder im Wiki.
Eine ausführliche Dokumentation zu diesem Dokument wird folgen.

\section{Kapitel 1}
Lorem ipsum dolor sit amet, consectetuer adipiscing elit. 
Aenean commodo ligula eget dolor. 
Aenean massa. 
Cum sociis natoque penatibus et magnis dis parturient montes, nascetur ridiculus mus. 
Donec quam felis, ultricies nec, pellentesque eu, pretium quis, sem. Nulla consequat massa quis enim. 
Donec pede justo, fringilla vel, aliquet nec, vulputate eget, arcu. In enim justo, rhoncus ut, imperdiet a, venenatis vitae, justo. 
Nullam dictum felis eu pede mollis pretium. 
Integer tincidunt. 
Cras dapibus. 
Vivamus elementum semper nisi. 

%? Quellen- und Literaturverzeichnis
% Quellen neigen dazu, zu breite Zeilen durch (z.B.) schlecht umgebrochene URLs zu erzeugen, die sich kaum vermeiden lassen. --> Ignorieren
\hfuzz=10pt
\printbibliography
\hfuzz=0.1pt

%? Anhang
%!	Anhang

\clearpage
\appendix
\clearpage

%! Section Befehl wird umgeschrieben, damit keine Überschriften mehr angezeigt werden
%!Kann falls Überschriften gewollt sind entfern werden oder erst später eingefügt
% Beginn 
\renewcommand{\section}[1]{
\par\refstepcounter{section}
\sectionmark{#1}
\addcontentsline{atoc}{section}{\protect\numberline{\thesection}#1}
\lohead{\textnormal{#1}}
} % Ende

%! Anhang 1
\section{Erster toller Anhang}
Hihi hier kommt eigentlich ein Anhangverzeichnis hin :D
\newpage

%! Anhang 2
\section{Inhalt der CD}
CD mit folgenden Inhalten:
\begin{itemize}
	\item dieses Dokument
	\item Latex Dateien
	\item Youtube-Video als Bonus
\end{itemize}
 \newpage

%! Eidestattliche Erklärung

%	Eidesstattliche Erklärung

\cleardoublepage 
\section{Ehrenwörtliche Erklärung}
\begin{bfseries}
	\begin{center}
		%! Hier muss noch Abstand rein!
		\Huge{Ehrenwörtliche Erklärung}\\[3cm]
	\end{center}
\end{bfseries}
			\begin{tabbing}
		Ich erkläre \= ehrenwörtlich,\\[1cm]
		1. 	\> dass ich meinen Praxisbeleg mit dem Thema:\\[1cm]
		   	\> Thema hier\\[1cm]
		ohne fremde Hilfe angefertigt habe;\\[1cm]
		
		2.	\> dass ich die Übernahme wörtlicher Zitate aus der Literatur sowie die\\ 		  
			\>Verwendung der Gedanken anderer Autoren an den entsprechenden\\
			\> Stellen innerhalb der Arbeit gekennzeichnet habe;\\[0.5cm]
		
		3.	\> dass ich meine Praxisbeleg bei keiner anderen Prüfung vorgelegt habe. \\[1cm]
		Ich bin mir bewusst, dass eine falsche Erklärung rechtliche Folgen haben wird. \\[2cm]
		\end{tabbing}
		
		 \begin{tabular}{p{8cm}l}
		  ----------------------------------- &  ----------------------------------- \\
		  Ort, Datum & Unterschrift  \\
		 \end{tabular}

%!##########################################

\end{document}
