%! Eigene Befehle zur erleichterten Nutzung
% Hilfsbefehle
\newcommand{\fontheightsvg}[1]{\includesvg[height=1.75ex, inkscapelatex=false]{#1}}
\newcommand{\dtfolder}{\fontheightsvg{vorlage/bilder/dirtree_folder}\hspace{0.1cm}}
\newcommand{\dtfile}{\fontheightsvg{vorlage/bilder/dirtree_file}\hspace{0.1cm}}
\newcommand{\dtusb}{\raisebox{.1em}{\fontheightsvg{vorlage/bilder/usb}\hspace{0.1cm}}}

% Umgebungen u.Ä.
\newcommand{\fn}[1]{\footnote{\hspace{0.5em}#1}}
%! #1 - Breite, #2 - Dateiname, #3 Caption, #4 - Label
\newcommand{\bild}[4][1.0]{\begin{figure}[H]
  \centering
  \includegraphics[width=#1\columnwidth]{bilder/#2}
  \caption{#3}
  \label{#4}
  \end{figure}}
\newcommand{\striche}[1]{\glqq #1\grqq{}}
%! #1 - Breite, #2 - Dateiname, #3 Caption, #4 - Label
\newcommand{\svg}[4][1.0]{\begin{figure}[H]
    \centering
    \includesvg[width=#1\columnwidth,inkscapelatex=false]{bilder/#2}
    \caption{#3}
    \label{#4}
    \end{figure}}
%! #1 - Dirtree, #2 - Caption, #3 - Label
\newcommand{\verzeichnis}[3]{\begin{figure}[H]
  % https://tex.stackexchange.com/a/99591/220899
  \renewcommand{\DTstyle}{\textrm\expandafter\raisebox{-0.7ex}}
  \centering
  \begin{varwidth}{\textwidth}
    \dirtree{#1}  
  \end{varwidth}
  \caption{#2}
  \label{#3}
  \end{figure}}
\newcommand{\logisch}[1]{$``#1``$}
\newcommand{\vglink}[2]{\fn{vgl.~\href{#1}{#1}~(#2)}}
% Muss statt \caption in die Umgebung wenn eine Fußnote verwendet werden soll, in Verbindung mit der Zeile darunter
\newcommand{\linkcaption}[1]{\caption[#1]{#1\footnotemark}}
% Muss unter die Umgebung, wenn eine Fußnote in der Umgebung verwendet werden soll
\newcommand{\vgcaption}[2]{\footnotetext{\hspace{0.5em}vgl.~\href{#1}{#1}~(#2)}}
\newcommand{\python}[1]{\mintinline{python}{#1}}
%! #1 - Formel, #2 - Legende, #3 - Caption, #4 - Label
\newcommand{\formula}[4]{\begin{formel}[H]
  \pretocmd{\captionbelow}{\onelinecaptionstrue}{}{}
  \KOMAoptions{captions=centeredbeside}
  \begin{captionbeside}[#3]{\textbf{#3}}[r]#1\end{captionbeside}
  \pretocmd{\captionbelow}{}{}{}#2\label{#4}\end{formel}}

%! Referenzierung
\newcommand{\literef}[1]{\emph{\hyperref[{#1}]{\autoref{#1} - \nameref{#1}}}}
\newcommand{\fullref}[1]{(\emph{\hyperref[{#1}]{siehe \autoref{#1} - \nameref{#1}}})}

% Kompatibilität zu alter "Architektur"
\let\aref\literef
\let\bref\literef
\let\cref\literef
\let\fref\literef
\let\sref\literef
\let\tref\literef
\let\litearef\literef
\let\litebref\literef
\let\litecref\literef
\let\litefref\literef
\let\litesref\literef
\let\litetref\literef
\let\fullaref\fullref
\let\fullbref\fullref
\let\fullcref\fullref
\let\fullfref\fullref
\let\fullsref\fullref
\let\fulltref\fullref

% automatische Ermittlung des ref-Typs nach https://tex.stackexchange.com/questions/33776/get-label-target-type
\makeatletter

% Helper macro to extract the type (section,subsection...) or the type name
% out of the label reference. Works with hyperref only.
% Argument #1 is a macro of form \def\...#1...\@nil{...}
% Argument #2 is the label reference, e.g. "sect:test"
\newcommand*\@autoref[2]{% \HyPsd@@@autoref from hyperref, modified
  \expandafter\ifx\csname r@#2\endcsname\relax
    ??%
  \else
    \expandafter\expandafter\expandafter\@@autoref
        \csname r@#2\endcsname{}{}{}{}\@nil#1\@nil
  \fi
}
\def\@@autoref#1#2#3#4#5\@nil#6\@nil{% \HyPsd@autorefname, modified
  #6#4.\@nil}% Argument #4 = type and number, e.g. "section.1" or "subsection.1.2"

% \reftype results in the type name, e.g. "section" or "figure".
% The starred variant will remove a star, if existent, i.e. "section*" will become "section"
\newcommand\reftype{%
  \@ifstar
    {\@autoref\@@reftype}%
    {\@autoref\@reftype}}
\def\@reftype#1.#2\@nil{#1}
\def\@@reftype#1.#2\@nil{\@@@reftype#1*\@nil}
\def\@@@reftype#1*#2\@nil{#1}

% \autoreftype results in the type prose name (plus space character),
% e.g. "section" in English or "Abschnitt" in German
% (like \autoref, but without number).
% \HyPsd@@autorefname is defined in the hyperref package.
\newcommand*\autoreftype[1]{\@autoref\HyPsd@@autorefname{#1}}

% An alternative version of \autoreftype without space at the end.
% Since the \space is hard coded inside \HyPsd@@autorefname we use our
% own version called \@autoreftype instead.
% Furthermore we offer a starred variant which will work with labels to
% \section* etc., too.
\renewcommand*\autoreftype{%
  \@ifstar
    {\@autoref\@@autoreftype}%
    {\@autoref\@autoreftype}}
\def\@autoreftype#1.#2\@nil{% = \HyPsd@@autorefname without \space
  \ltx@IfUndefined{#1autorefname}%
    {\ltx@IfUndefined{#1name}%
      {}%
      {\csname#1name\endcsname}}%
    {\csname#1autorefname\endcsname}}
\def\@@autoreftype#1.#2\@nil{%
  \expandafter\@autoreftype\@@@reftype#1*\@nil.\@nil}

\makeatother

% Bezeichnungen Unterabschnitt und Unterunterabschnitt durch Abschnitt ersetzen, auskommentieren, falls diese Bezeichner erwünscht sind
\AtBeginDocument{%
  \renewcommand*{\subsectionautorefname}{\sectionautorefname}%
  \renewcommand*{\subsubsectionautorefname}{\sectionautorefname}%
}
