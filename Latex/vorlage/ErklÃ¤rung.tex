%  Eidesstattliche Erklärung
%! Dies ist eine zur Nutzung mit LaTeX angepasste Version der in Anhang 6 der Hinweise zur Anfertigung
%! wissenschaftlicher Arbeiten an der Staatlichen Studienakademie Glauchau vorgegebenen Zustimmung
%! zur Plagiatsprüfung für Praxisbelege.
\cleardoublepage
\lohead{\textnormal{Eidesstattliche Erklärung}}
\rohead{}
    \vspace*{1cm}
    \begin{center}
        \huge\textbf{Eidesstattliche Erklärung}\\
    \end{center}
    \vspace*{1cm}
    \normalsize
    \ifthenelse{\isundefined{\autorzwei}}{Ich erkläre}{Wir erklären} an Eides statt,

    dass \ifthenelse{\isundefined{\autorzwei}}{ich}{wir} die vorliegende Arbeit selbständig und nur unter Verwendung der angegebenen Quellen und Hilfsmittel angefertigt \ifthenelse{\isundefined{\autorzwei}}{habe}{haben}.

    Die aus fremden Quellen direkt oder indirekt übernommenen Stellen sind als solche kenntlich gemacht.

    Die Zustimmung des/der beteiligten Unternehmen/s zur Verwendung betrieblicher Unterlagen habe ich eingeholt.

    Die Arbeit wurde bisher in gleicher oder ähnlicher Form weder veröffentlicht noch einer anderen Prüfungsbehörde/-stelle vorgelegt.

    \vspace{2cm}
    % Autoren Namen einfügen
    \autoreins\ifthenelse{\isundefined{\autorzwei}}{}{, \autorzwei\ifthenelse{\isundefined{\autordrei}}{}{, \autordrei\ifthenelse{\isundefined{\autorvier}}{}{, \autorvier}}}\\
    \noindent\rule{0.35\columnwidth}{0.4pt}\\
    \ifthenelse{\isundefined{\autorzwei}}{Name Verfassender}{Namen Verfassende}
    \vspace{2cm}
    
    Glauchau, \abgabedatum\newline\noindent\rule{0.35\columnwidth}{0.4pt}\hspace{0.05\columnwidth}\rule{0.6\columnwidth}{0.4pt}\\
    Ort, Datum\hspace{0.27\columnwidth}\ifthenelse{\isundefined{\autorzwei}}{Unterschrift Verfassender}{Unterschriften Verfassende}

\newpage

  \lohead{\textnormal{Zustimmung Plagiatsprüfung}}
  \rohead{}
      \vspace*{2mm}

      \begin{minipage}{0.5\columnwidth}
          \includesvg[width=\columnwidth]{vorlage/bilder/ba-gc-logo}
          % Alternativ mit PNG Logo, falls Inkscape nicht installiert werden, bzw. nicht der PATH-Variable hinzugefügt werden soll.
          %! Die SVG-Version sieht im Druck deutlich besser aus.
          %\includegraphics[width=\columnwidth]{vorlage/bilder/ba-gc-logo}
      \end{minipage}
      \begin{minipage}{0.45\columnwidth}
          \begin{flushright}
              {\small nach 4BA-F.219\\}
          \end{flushright}
      \end{minipage}
      \vspace*{2mm}

      \begin{center}
          \textbf{\huge{Erklärung zur Prüfung wissenschaftlicher Arbeiten}}
      \end{center}

      Die Bewertung wissenschaftlicher Arbeiten erfordert die Prüfung auf Plagiate. Die hierzu von der Staatlichen Studienakademie Glauchau eingesetzte Prüfungskommission nutzt sowohl eigene Software als auch diesbezügliche Leistungen von Drittanbietern. Dies erfolgt gemäß \href{https://www.revosax.sachsen.de/vorschrift/1672-Saechsisches-Datenschutzgesetz#p7}{§ 7 des Gesetzes zum Schutz der informationellen Selbstbestimmung im Freistaat Sachsen (Sächsisches Datenschutzgesetz - SächsDSG)} vom 25. August 2003 (Rechtsbereinigt mit Stand vom 31. Juli 2011) im Sinne einer Datenverarbeitung im Auftrag.

      Der Studierende bevollmächtigt die Mitglieder der Prüfungskommission hiermit zur Inanspruchnahme o. g. Dienste. In begründeten Ausnahmefällen kann der Datenschutzbeauftragte der Berufsakademie Sachsen sowohl vom Verfasser der wissenschaftlichen Arbeit als auch von der Prüfungskommission in den Entscheidungsprozess einbezogen werden.

      \ifthenelse{\isundefined{\autorzwei}}{%
        \arrayrulewidth=0.5pt
        \begin{table}[H]
            \centering
            \begin{tabularx}{\columnwidth}{|p{3.2cm}|X|}
                \hline
                Name:             & \autoreins\\
                \hline
                Matrikelnummer:   & \matnumeins\\
                \hline
                Studiengang:      & \studiengang\\
                \hline
                Titel der Arbeit: & \titel\\
                \hline
                Datum:            & \abgabedatum\\
                \hline
                Unterschrift:     & \\
                                  & \\
                \hline
            \end{tabularx}
        \end{table}
      }{\ifthenelse{\isundefined{\autordrei}}{%
        \begin{table}[H]
          \centering
          \newcolumntype{Y}{>{\centering\arraybackslash}X}
          \begin{tabularx}{\columnwidth}{|X|Y|Y|}
              \hline
              Namen:            & \autoreins  & \autorzwei  \\
              \hline
              Matrikelnummern:  & \matnumeins & \matnumzwei \\
              \hline
              Studiengang:      & \multicolumn{2}{c|}{\studiengang}\\
              \hline
              Titel der Arbeit: & \multicolumn{2}{c|}{\titel}\\
                              %& \multicolumn{2}{c|}{zweite Zeile, falls nötig}\\
              \hline
              Datum:            & \multicolumn{2}{c|}{\abgabedatum}\\
              \hline
              Unterschriften:   &             &\\
                                &             &\\
              \hline
          \end{tabularx}
        \end{table}
      }{\ifthenelse{\isundefined{\autorvier}}{%
        \begin{table}[H]
          \centering
          \newcolumntype{Y}{>{\centering\arraybackslash}X}
          \begin{tabularx}{\columnwidth}{|X|Y|Y|Y|}
              \hline
              Namen:            & \autoreins  & \autorzwei  & \autordrei \\
              \hline
              Matrikelnummern:  & \matnumeins & \matnumzwei & \matnumdrei \\
              \hline
              Studiengang:      & \multicolumn{3}{c|}{\studiengang}\\
              \hline
              Titel der Arbeit: & \multicolumn{3}{c|}{\titel}\\
                              %& \multicolumn{3}{c|}{zweite Zeile, falls nötig}\\
              \hline
              Datum:            & \multicolumn{3}{c|}{\abgabedatum}\\
              \hline
              Unterschriften:   &             &             & \\
                                &             &             &\\
              \hline
          \end{tabularx}
        \end{table}
      }{
        \begin{table}[H]
          \centering
          \newcolumntype{Y}{>{\centering\arraybackslash}X}
          \begin{tabularx}{\columnwidth}{|X|Y|Y|Y|Y|}
              \hline
              Namen:            & \autoreins  & \autorzwei  & \autordrei  & \autorvier\\
              \hline
              Mat.-Num.:        & \matnumeins & \matnumzwei & \matnumdrei & \matnumvier\\
              \hline
              Studiengang:      & \multicolumn{4}{c|}{\studiengang}\\
              \hline
              Titel der Arbeit: & \multicolumn{4}{c|}{\titel}\\
                                  %& \multicolumn{4}{c|}{zweite Zeile, falls nötig}\\
              \hline
              Datum:            & \multicolumn{4}{c|}{\abgabedatum}\\
              \hline
              Unterschriften:   &              &             &            &\\
                                &              &             &            &\\
              \hline
          \end{tabularx}
        \end{table}
      }
      }
    }
