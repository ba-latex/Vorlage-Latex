%Alle Kapitel beginnen mit der Hauptüberschrift
\chapter{Einleitung}
Diese PDF wurde mit der Vorlage erstellt, um die Funktion und Formatierung dieser zu zeigen.

Die Vorlage\onlinezitat{Vorlage} ist ein Gemeinschaftsprojekt im Rahmen unseres Studiums.
Der Docker-Container\onlinezitat{HILLE2021} gehört dazu.
Die Vorlage richtet sich weitestgehend nach dem Dokument \ac{HAWA}\onlinezitat{HAWA} der \href{https://www.ba-glauchau.de/}{Staatlichen Studienakademie Glauchau}.

Weitere Hinweise befinden sich in der README.md oder im \href{https://github.com/DSczyrba/Vorlage-Latex/wiki}{Wiki}.
Eine ausführliche Dokumentation zu diesem Dokument wird folgen.

\chapter{Beispiele}
\label{sec:beispiele}
\section{Pixelgrafiken}
Siehe Quelltext.\fn{Dort wird das Kommando \striche{bild} aufgeführt. In dieser Vorlage sind allerdings keine Bilder (im vorgesehenen Ordner) enthalten.}
%\bild[0.5]{dateiname-ohne-endung}{Text zu einem Bild}{label1}
\section{Vektorgrafiken}
Siehe Quelltext.
%\svg[0.5]{dateiname-ohne-endung}{Text zu einer \ac{SVG}-Datei}{label2}
\section{Programmcode}
\begin{code}
    \begin{minted}{python}
        import asdf
        from foo import bar
        
        def yolo():
            return "glhf"
        
        class Wooo(Foo):
            def __init__(self, boo):
                self.doo = boo
                moo = 1 + 2 +3
    \end{minted}
    \caption[Beispielcode]{Beispielcode}
    \label{code:example}
\end{code}

An dieser Stelle wird noch ein einzeiliger Code eingefügt, der keine Zeilennummerierung erhalten soll.
Dies kann genutzt werden, wenn man nur kurz auf eine Funktion eingehen möchte und diese nicht im Quellcodeverzeichnis erscheinen soll.
\mint{python}|print("Hallo Vorlage")|

Ein weiterer Code um zu schauen, ob danach die Zeilennummerierung wieder funktioniert.

\begin{code}
  \begin{minted}{python}
      class Wooo(Foo):
          def __init__(self, boo):
              self.doo = boo
              moo = 1 + 2 +3
  \end{minted}
  \linkcaption{Beispielcode}
  \label{code:example2}
\end{code}
\vgcaption{https://link-wo-es-den-code-gibt.de}{06.07.2022}
\section{Ordnerstruktur}
    In diesem Abschnitt wird eine Ordnerstruktur in \literef{beispielbaum} gezeigt.
    Ordnerstrukturen können im Quellcode definiert werden.
    Die Symbole für Ordner und Dateien können, auf Wunsch, ausgetauscht oder erweitert werden, um verschiedene Dateitypen abzubilden.
    \verzeichnis{%
          .1 \dtfolder Vorlage-Latex. .2 \dtfile HINWEISE.md.
          .2 \dtfile LICENSE.
          .2 \dtfile README.md.
          .2 \dtfile sortieren.py.
          .2 \dtfolder Latex.
            .3 \dtfolder bilder.
              .4 \dtfile firmenlogo.svg.
            .3 \dtfolder inhalt.
              .4 \dtfile Abkürzungen.tex.
              .4 \dtfile Anhang.tex.
            .3 \dtfolder light.
              .4 \dtfile main.tex.
              .4 \dtfile README.md.
              .4 \dtfile vorlage\_light.tex.
            .3 \dtfile literatur.bib.
            .3 \dtfile main.tex.
            .3 \dtfile main\_abstract.tex.
            .3 \dtfile metadaten.sty.
    }{Ein Verzeichnis-Baum}{beispielbaum}

    \section{Formeln}
    Formeln werden mittels \striche{\textbackslash formula} in die Arbeit eingebunden.
    Die Beschriftungen befinden sich, wie es die HAWA\onlinezitat[Abs. 3.3.3.7]{HAWA} verlangt, rechts neben der Formel.
    \formula{$\Delta p_{WZ}=\Delta p\cdot\dfrac{\dot{V}^2_S}{\dot{V}^2_G}$}{%
    $\dot{V}^2_S = $ Spitzendurchfluss $\left[ m^3/h\right]$\\
    $\dot{V}^2_G = $ maximaler Durchfluss im Wasserzähler $\left[ m^3/h\right]$\\
    $\Delta p = $ Druckverlust bei $V_{max} \left[bar\right]$}{Druckverlust}{formel:ohm}

    \section{Überschriften}
    \subsection{in verschiedenen}
    \subsubsection{Größen}
    Kapitelüberschriften werden per \emph{\textbackslash chapter}, Abschnitte per \emph{\textbackslash section}, Unterabschnitte per \emph{\textbackslash subsection} und Unterunterabschnitte per \emph{\textbackslash subsubsection} eingefügt.

\chapter{Test-/Dokutabelle}
Diese Tabelle dient hauptsächlich zum Testen der einzelnen Kommandos, sowie als minimales Beispiel für eine \emph{tabularx}-Tabelle:
\hbadness=10000 %"Tabelle 1" in Spalte Beispiel erzeugt sonst eine Warnung
\begin{table}
\begin{tabularx}{\columnwidth}{|p{3cm}|X|p{.2\columnwidth}|}
\hline
Gegenstand & Beispiel & Befehl \\
\hline
Kurzer Verweis & \literef{sec:beispiele} & \emph{\textbackslash literef}\\
\hline
Langer Verweis & \fullref{beispielbaum} & \emph{\textbackslash fullref}\\
\hline
Verweis ohne Objektname & \autoref{beispieltabelle} & \emph{\textbackslash autoref}\\
\hline
\multicolumn{2}{|c|}{verbundene Spalten mit zentriertem Text} & \emph{\textbackslash multicolumn} \\
\hline
Zeile 1 & \multirow{2}{\hsize}{verbundene Zeilen inklusive automatischer Zeilenumbrüche} & \emph{\textbackslash multirow} \\
\cline{1-1}\cline{3-3}
Zeile 2 & & mit \emph{\textbackslash hsize}\\
\hline
\end{tabularx}
\caption{Beispieltabelle}
\label{beispieltabelle}
\end{table}
\hbadness=1000

In diesem Satz wird eine Quelle mit nur einem Autor zitiert\onlinezitat{HAWA}, eine Quelle mit drei Autoren\vgonlinezitat{Vorlage} wird sinngemäß zitiert und eine Quelle mit vier Autoren\onlinezitat[Abs.~2.3]{rfc3596}, welche als \emph{Autor 1; u.a.} dargestellt werden sollte, existiert ebenfalls. Es folgen ein Buch-Zitat mit Seitenangabe\zitat[9]{KOHM2020} und ein sinngemäßes Zitat aus einer unveröffentlichten Quelle\vguvzitat{UV}.
