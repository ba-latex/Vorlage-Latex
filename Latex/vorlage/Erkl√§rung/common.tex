% Diese Datei generiert die Ehrenwörtliche/Eidesstattliche Erklärung.
%? Der Text wird anhand der in metadaten.sty (de)aktivierten Kommandos verändert.
%? Es wird zwischen Einem Autor ohne \jahr-Kommando (Praxisbeleg), einem Autor mit \jahr-Kommando (Bachelorthesis) und mehreren Autoren (Projektarbeit) unterschieden.
%? Die Tabelle für die Unterschrift(en) wird jeweils aus einer Datei je nach Anzahl der Autoren nachgeladen.
%! Falls der Titel der Arbeit zu lang für eine Zeile ist, werden in diesen ausgelagerten Dateien bereits (auskommentierte) Zeilen für eine zweite Titelseite bereitgestellt.
\cleardoublepage
\lohead{\textnormal{Ehrenwörtliche Erklärung}}
\rohead{}
    \vspace*{1cm}
    \begin{center}
        \huge\textbf{Ehrenwörtliche Erklärung}\\
    \end{center}
    \vspace*{1cm}
    \normalsize
    \ifthenelse{\isundefined{\autorzwei}}{Ich erkläre}{Wir erklären} hiermit ehrenwörtlich,

    \begin{enumerate}
        \vspace{1cm}
        \item dass \ifthenelse{\isundefined{\autorzwei}}{ich \ifthenelse{\isundefined{\jahr}}{meinen Praxisbeleg}{meine Bachelorthesis}}{wir unsere Projektarbeit} mit dem Thema:\\

        \textbf{\titel }\\

        ohne fremde Hilfe angefertigt habe\ifthenelse{\isundefined{\autorzwei}}{}{n},
        \item dass \ifthenelse{\isundefined{\autorzwei}}{ich}{wir} die Übernahme wörtlicher Zitate aus der Literatur sowie die\\
        Verwendung der Gedanken anderer Autoren an den entsprechenden\\
        Stellen innerhalb der Arbeit gekennzeichnet habe\ifthenelse{\isundefined{\autorzwei}}{}{n} und
        \item dass \ifthenelse{\isundefined{\autorzwei}}{ich \ifthenelse{\isundefined{\jahr}}{meinen Praxisbeleg}{meine Bachelorthesis}}{wir unsere Projektarbeit} bei keiner anderen Prüfung vorgelegt habe\ifthenelse{\isundefined{\autorzwei}}{}{n}.\\[1,5cm]
    \end{enumerate}
    \ifthenelse{\isundefined{\autorzwei}}{Ich bin mir}{Wir sind uns} bewusst, dass eine falsche Erklärung rechtliche Folgen haben wird.\\[1,5cm]

    Glauchau, \abgabedatum\newline\noindent\rule{0.35\columnwidth}{0.4pt}\hspace{0.05\columnwidth}\rule{0.6\columnwidth}{0.4pt}\\
    Ort, Datum\hspace{0.27\columnwidth}Unterschriften


    \newpage
\lohead{\textnormal{Zustimmung Plagiatsprüfung}}
\rohead{}
    
    \vspace*{2mm}

    \begin{minipage}{0.5\columnwidth}
        \includesvg[width=\columnwidth]{vorlage/bilder/ba-gc-logo}
        % Alternativ mit PNG Logo, falls Inkscape nicht installiert werden, bzw. nicht der PATH-Variable hinzugefügt werden soll.
        %! Die SVG-Version sieht im Druck deutlich besser aus.
        %\includegraphics[width=\columnwidth]{vorlage/bilder/ba-gc-logo}
    \end{minipage}
    \begin{minipage}{0.45\columnwidth}
        \begin{flushright}
            {\small nach 4BA-F.219\\}
        \end{flushright}
    \end{minipage}
    \vspace*{2mm}

    \begin{center}
        \textbf{\huge{Erklärung zur Prüfung wissenschaftlicher Arbeiten}}
    \end{center}

    Die Bewertung wissenschaftlicher Arbeiten erfordert die Prüfung auf Plagiate. Die hierzu von der Staatlichen Studienakademie Glauchau eingesetzte Prüfungskommission nutzt sowohl eigene Software als auch diesbezügliche Leistungen von Drittanbietern. Dies erfolgt gemäß \href{https://www.revosax.sachsen.de/vorschrift/1672-Saechsisches-Datenschutzgesetz#p7}{§ 7 des Gesetzes zum Schutz der informationellen Selbstbestimmung im Freistaat Sachsen (Sächsisches Datenschutzgesetz - SächsDSG)} vom 25. August 2003 (Rechtsbereinigt mit Stand vom 31. Juli 2011) im Sinne einer Datenverarbeitung im Auftrag.

    Die Studierenden bevollmächtigen die Mitglieder der Prüfungskommission hiermit zur Inanspruchnahme o. g. Dienste. In begründeten Ausnahmefällen kann der Datenschutzbeauftragte der Berufsakademie Sachsen sowohl von den Verfassern der wissenschaftlichen Arbeit als auch von der Prüfungskommission in den Entscheidungsprozess einbezogen werden.

    \arrayrulewidth=0.5pt
