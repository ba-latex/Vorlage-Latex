
%	Eidesstattliche Erklärung

\cleardoublepage 
\section{Ehrenwörtliche Erklärung}
\vspace*{1cm}
\begin{center}
\huge\textbf{Ehrenwörtliche Erklärung}\\
\end{center}
\vspace*{1cm}
\normalsize
Wir erklären hiermit ehrenwörtlich,

\begin{enumerate}
	\vspace{1cm}
	\item dass wir unsere Belegarbeit mit dem Thema:\\
	
	\textbf{\titel }\\

	ohne fremde Hilfe angefertigt haben,
	\item dass wir die Übernahme wörtlicher Zitate aus der Literatur sowie die\\ 		  
	Verwendung der Gedanken anderer Autoren an den entsprechenden\\
	Stellen innerhalb der Arbeit gekennzeichnet haben und
	\item dass wir unsere Belegarbeit bei keiner anderen Prüfung vorgelegt haben.\\[1,5cm]
\end{enumerate}
Wir sind uns bewusst, dass eine falsche Erklärung rechtliche Folgen haben wird.\\[1,5cm]
		
\vfill

Glauchau, \abgabedatum\newline\noindent\rule{0.35\columnwidth}{0.4pt}\hspace{0.05\columnwidth}\rule{0.6\columnwidth}{0.4pt}\\
Ort, Datum\hspace{0.27\columnwidth}Unterschriften

{\footnotesize Dies ist eine zur Nutzung mit mehreren Autoren und \LaTeX\ angepasste Version der in \href{https://www.ba-glauchau.de/fileadmin/glauchau/waehrend-des-studium/dokumente/pruefungen/4BA-F.207_Hinweise_zur_Anfertigung_wissenschaftlicher_Arbeiten.pdf}{Anhang 4 der Hinweise zur Anfertigung wissenschaftlicher Arbeiten an der Staatlichen Studienakademie Glauchau vorgegebenen ehrenwörtlichen Erklärung.}}


\newpage
\section{Zustimmung Plagiatsprüfung}

\vspace*{2mm}

\begin{minipage}{0.5\columnwidth}
\includesvg[width=\columnwidth]{bilder/ba_glauchau_logo.svg}
%alternativ mit PNG Logo, falls Inkscape nicht installiert werden, bzw. nicht der PATH-Variable hinzugefügt werden soll
%MERKE: die SVG-Version sieht im Druck deutlich besser aus
%\includegraphics[width=\columnwidth]{bilder/ba-gc-logo.png}
\end{minipage}
\begin{minipage}{0.45\columnwidth}
\begin{flushright}
{\small nach 4BA-F.219\\}
\end{flushright}
\end{minipage}
\vspace*{2mm}

\begin{center}\textbf{\huge{Erklärung zur Prüfung wissenschaftlicher Arbeiten}}\end{center}

Die Bewertung wissenschaftlicher Arbeiten erfordert die Prüfung auf Plagiate. Die hierzu von der Staatlichen Studienakademie Glauchau eingesetzte Prüfungskommission nutzt sowohl eigene Software als auch diesbezügliche Leistungen von Drittanbietern. Dies erfolgt gemäß \href{https://www.revosax.sachsen.de/vorschrift/1672-Saechsisches-Datenschutzgesetz#p7}{§ 7 des Gesetzes zum Schutz der informationellen Selbstbestimmung im Freistaat Sachsen (Sächsisches Datenschutzgesetz - SächsDSG)} vom 25. August 2003 (Rechtsbereinigt mit Stand vom 31. Juli 2011) im Sinne einer Datenverarbeitung im Auftrag.

Die Studierenden bevollmächtigen die Mitglieder der Prüfungskommission hiermit zur Inanspruchnahme o. g. Dienste. In begründeten Ausnahmefällen kann der Datenschutzbeauftragte der Berufsakademie Sachsen sowohl von den Verfassern der wissenschaftlichen Arbeit als auch von der Prüfungskommission in den Entscheidungsprozess einbezogen werden.

\arrayrulewidth=0.5pt

\begin{table}[H]
\centering
\newcolumntype{Y}{>{\centering\arraybackslash}X}
\begin{tabularx}{\columnwidth}{|X|Y|Y|Y|}
\hline
Namen: & \autoreins & \autorzwei & \autordrei \\
\hline
Matrikelnummern: & \matnumeins & \matnumzwei & \matnumdrei \\
\hline
Studiengang: & \multicolumn{3}{c|}{\studiengang}\\
\hline
Titel der Arbeit: & \multicolumn{3}{c|}{Schaltung eines binären}\\
& \multicolumn{3}{c|}{3-Bit-Additions-Subtraktionswerks}\\
\hline
Datum: & \multicolumn{3}{c|}{\abgabedatum}\\
\hline
Unterschriften: & & & \\
&&&\\
\hline
\end{tabularx}
\end{table}

\vfill

{\footnotesize Dies ist eine zur Nutzung mit mehreren Autoren und \LaTeX\ angepasste Version der in \href{https://www.ba-glauchau.de/fileadmin/glauchau/waehrend-des-studium/dokumente/pruefungen/4BA-F.207_Hinweise_zur_Anfertigung_wissenschaftlicher_Arbeiten.pdf}{Anhang 6 der Hinweise zur Anfertigung wissenschaftlicher Arbeiten an der Staatlichen Studienakademie Glauchau vorgegebenen Zustimmung zur Plagiatsprüfung.}}