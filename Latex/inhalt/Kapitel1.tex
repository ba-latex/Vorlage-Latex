\usemintedstyle{colorful} %Festlegen des Minted Styles

\section{Einleitung}
  Diese PDF wurde mit der Vorlage erstellt, um die Funktion und Formatierung dieser zu zeigen.

  Die Vorlage\onlinezitat{SCZYRBA2020} ist ein Gemeinschaftsprojekt im Rahmen unseres Studiums.
  Der Docker-Container\onlinezitat{HILLE2021} gehört dazu.
  Die Vorlage richtet sich weitestgehend nach dem Dokument \ac{HAWA}\onlinezitat{HAWA} der \href{https://www.ba-glauchau.de/}{Staatlichen Studienakademie Glauchau}.

  Weitere Hinweise befinden sich in der README.md oder im \href{https://github.com/DSczyrba/Vorlage-Latex/wiki}{Wiki}.
  Eine ausführliche Dokumentation zu diesem Dokument wird folgen.

\section{Dokumentation der Vorlage}
  \subsection{Einbinden von Grafiken}
    \subsubsection{Pixelgrafiken}
      Es ist sowohl möglich Pixelgrafiken wie in diesem Kapitel beschrieben 
      oder auch wie im nächsten Kapitel \ref{sec:vectorgrafiken} beschrieben Vektorgrafiken einzubinden.

      Prinzipiell gibt es zwei Möglichkeiten eine Grafik einzubinden.
      Zum einen über eine vordefinierte Funktion namens \mintinline{latex}{\bild{}}. 
      Diese benutzt im Endeffekt die zweite Variante ist durch den kürzeren Befehl allerdings deutlich einfacher zu nutzen.
            
      \begin{code}[H]
        \begin{minted}[linenos, breaklines, frame=none, numbers=left, autogobble, numbersep=5pt]{latex}
          \bild[0.5]{ba-gc-logo}{Text zu einem Bild}{fig:ba-gc-logo}
          % \bild ruft die Funktion auf
          % 0.5 steht für die Breite des Bildes (0.0 bis 1.0)
          % ba-gc-logo gibt den Dateinamen des Bildes im Ordner bilder an
          % label beschreibt den Namen für spätere \refs{}
        \end{minted}
        \caption{Grafiken mittels Bild-Funktion einbinden}
        \label{code:bild-einfuegen1}
        \end{code}

        Zum anderen über die \LaTeX-Standardvariante einer Gleitumgebung, im Folgenden \striche{Environment} genannt.

        \begin{code}[H]
          \begin{minted}[linenos, breaklines, frame=none, numbers=left, autogobble, numbersep=5pt]{latex}
            \begin{figure}[H]
              \centering 
              \includegraphics[width=0.5\columnwidth]{bilder/ba-gc-logo}
              \caption{Bildunterschrift}
              \label{fig:ba-gc-logo}
            \end{figure}
          \end{minted}
          \caption{Grafiken mittels Environments einfügen}
          \label{code:bild-einfuegen2}
        \end{code}

        In der folgenden Abbildung wird die in den Codes beschriebene Grafik eingebunden.
        \bild[0.5]{ba-gc-logo}{Text zu einem Bild}{label1}

  \subsubsection{Vektorgrafiken}
    \label{sec:vectorgrafiken}
    Vektorgrafiken verhalten sich prinzipiell ähnlich wie im vorherigen Kapitel beschriebene Pixelgrafiken.
    Dennoch sollen hier aus Gründen der Vollständigkeit beide Varianten gezeigt werden.

    \begin{code}[H]
      \begin{minted}[linenos, breaklines, frame=none, numbers=left, autogobble, numbersep=5pt]{latex}
        \svg[0.5]{ba_glauchau_logo}{Text zu einer SVG-Datei}{fig:ba-gc-logo}
        % \svg ruft die Funktion auf
        % 0.5 steht für die Breite des Bildes (0.0 bis 1.0)
        % ba_glauchau_logo gibt den Dateinamen des Bildes im Ordner bilder an
        % label beschreibt den Namen für spätere \refs{}
      \end{minted}
      \caption{SVG-Grafiken mittels Bild-Funktion einbinden}
      \label{code:svg-einfuegen1}
      \end{code}

      \begin{code}[H]
        \begin{minted}[linenos, breaklines, frame=none, numbers=left, autogobble, numbersep=5pt]{latex}
          \begin{figure}[H]
            \centering
            \includesvg[width=0.5\columnwidth,inkscapelatex=false]{bilder/ba}
            \caption{Bildunterschrift}
            \label{fig:ba-gc-logo}
          \end{figure}
        \end{minted}
        \caption{SVG-Grafiken mittels Environments einfügen}
        \label{code:bildsvg-einfuegen2}
      \end{code}

      In der folgenden Abbildung wird die in den Codes beschriebene SVG-Grafik eingebunden.
      \svg[0.5]{ba_glauchau_logo}{Text zu einer \ac{SVG}-Datei}{label2}

\section{Kapitel 1}
  \blindtext
