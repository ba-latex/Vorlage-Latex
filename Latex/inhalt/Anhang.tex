%!	Anhang

\clearpage
\appendix
\clearpage

%! Section Befehl wird umgeschrieben, um Überschriften zu verbergen
%! Kann, falls Überschriften gewollt sind, entfernt oder erst später eingefügt werden.
% Beginn
\renewcommand{\section}[1]{
\par\refstepcounter{section}
\sectionmark{#1}
\addcontentsline{atoc}{section}{\bfseries\protect\numberline{\thesection}{\mdseries#1}}
\lohead{\textnormal{#1}}
}
% Ende

%! Anpassung der Darstellung von Abbildungen im Anhang
%! Eine Variante auskommentieren
%? Möglichkeit 1: ohne Nummerierung und ohne "Abbildung"
%\renewcommand{\bild}[3][1.0]{\begin{figure}[H]
%	\centering
%	\includegraphics[width=#1\columnwidth]{bilder/#2}
%	\caption*{#3}
%	\label{fig:#3}
%	\end{figure}}

%? Möglichkeit 2: mit Nummerierung und "Abbildung", aber nicht im Abbildungsverzeichnis
\renewcommand{\bild}[3][1.0]{\begin{figure}[H]
	\centering
	\includegraphics[width=#1\columnwidth]{bilder/#2}
	\caption[]{#3}
	\label{fig:#3}
	\end{figure}}

%! Anhang 1
\section{Erster Anhang}
Der erste Anhang der Arbeit.
\clearpage

%! Anhang 2
\section{Inhalt der CD}
CD mit folgenden Inhalten:
\begin{itemize}
	\item dieses Dokument
	\item \LaTeX -Dateien
	\item \href{https://www.youtube.com/watch?v=dQw4w9WgXcQ}{YouTube-Video} als Bonus
\end{itemize}
\clearpage

%  Eidestattliche Erklärung und Erklärung zur Prüfung wissenschaftlicher Arbeiten
%! hier zwischen Version für einen oder mehrere Autoren umschalten

%	Eidesstattliche Erklärung

\cleardoublepage 
\section{Ehrenwörtliche Erklärung}
\vspace*{1cm}
\begin{center}
\huge\textbf{Ehrenwörtliche Erklärung}\\
\end{center}
\vspace*{1cm}
\normalsize
Ich erkläre hiermit ehrenwörtlich,

\begin{enumerate}
	\vspace{1cm}
	\item dass ich meinen Praxisbeleg mit dem Thema:\\
	
	\textbf{\titel }\\

	ohne fremde Hilfe angefertigt habe,
	\item dass ich die Übernahme wörtlicher Zitate aus der Literatur sowie die\\ 		  
	Verwendung der Gedanken anderer Autoren an den entsprechenden\\
	Stellen innerhalb der Arbeit gekennzeichnet habe und
	\item dass ich meinen Praxisbeleg bei keiner anderen Prüfung vorgelegt habe.\\[1,5cm]
\end{enumerate}
Ich bin mir bewusst, dass eine falsche Erklärung rechtliche Folgen haben wird.\\[1,5cm]
			
\vfill

Glauchau, \abgabedatum \newline\noindent\rule{0.35\columnwidth}{0.4pt}\hspace{0.05\columnwidth}\rule{0.6\columnwidth}{0.4pt}\\
Ort, Datum\hspace{0.27\columnwidth}Unterschrift

{\footnotesize Dies ist eine zur Nutzung mit \LaTeX\ angepasste Version der in \href{https://www.ba-glauchau.de/fileadmin/glauchau/waehrend-des-studium/dokumente/pruefungen/4BA-F.207_Hinweise_zur_Anfertigung_wissenschaftlicher_Arbeiten.pdf}{Anhang 4 der Hinweise zur Anfertigung wissenschaftlicher Arbeiten an der Staatlichen Studienakademie Glauchau vorgegebenen ehrenwörtlichen Erklärung.}}

\newpage
\section{Zustimmung Plagiatsprüfung}

\vspace*{2mm}

\begin{minipage}{0.5\columnwidth}
\includesvg[width=\columnwidth]{bilder/ba_glauchau_logo.svg}
%alternativ mit PNG Logo, falls Inkscape nicht installiert werden, bzw. nicht der PATH-Variable hinzugefügt werden soll
%MERKE: die SVG-Version sieht im Druck deutlich besser aus
%\includegraphics[width=\columnwidth]{bilder/ba-gc-logo.png}
\end{minipage}
\begin{minipage}{0.45\columnwidth}
\begin{flushright}
{\small nach 4BA-F.219\\}
\end{flushright}
\end{minipage}
\vspace*{2mm}

\begin{center}\textbf{\huge{Erklärung zur Prüfung wissenschaftlicher Arbeiten}}\end{center}

Die Bewertung wissenschaftlicher Arbeiten erfordert die Prüfung auf Plagiate. Die hierzu von der Staatlichen Studienakademie Glauchau eingesetzte Prüfungskommission nutzt sowohl eigene Software als auch diesbezügliche Leistungen von Drittanbietern. Dies erfolgt gemäß \href{https://www.revosax.sachsen.de/vorschrift/1672-Saechsisches-Datenschutzgesetz#p7}{§ 7 des Gesetzes zum Schutz der informationellen Selbstbestimmung im Freistaat Sachsen (Sächsisches Datenschutzgesetz - SächsDSG)} vom 25. August 2003 (Rechtsbereinigt mit Stand vom 31. Juli 2011) im Sinne einer Datenverarbeitung im Auftrag.

Der Studierende bevollmächtigt die Mitglieder der Prüfungskommission hiermit zur Inanspruchnahme o. g. Dienste. In begründeten Ausnahmefällen kann der Datenschutzbeauftragte der Berufsakademie Sachsen sowohl vom Verfasser der wissenschaftlichen Arbeit als auch von der Prüfungskommission in den Entscheidungsprozess einbezogen werden.

\arrayrulewidth=0.5pt
\begin{table}[H]
\centering
\begin{tabularx}{\columnwidth}{|p{3cm}|X|}
\hline
Name: & \autoreins\\
\hline
Matrikelnummer: & \matnumeins\\
\hline
Studiengang: & \studiengang\\
\hline
Titel der Arbeit: &\titel\\
\hline
Datum: & \abgabedatum\\
\hline
Unterschrift: & \\
& \\
\hline
\end{tabularx}
\end{table}

\vfill

{\footnotesize Dies ist eine zur Nutzung mit \LaTeX\ angepasste Version der in \href{https://www.ba-glauchau.de/fileadmin/glauchau/waehrend-des-studium/dokumente/pruefungen/4BA-F.207_Hinweise_zur_Anfertigung_wissenschaftlicher_Arbeiten.pdf}{Anhang 6 der Hinweise zur Anfertigung wissenschaftlicher Arbeiten an der Staatlichen Studienakademie Glauchau vorgegebenen Zustimmung zur Plagiatsprüfung.}}
%
%	Eidesstattliche Erklärung

\cleardoublepage 
\section{Ehrenwörtliche Erklärung}
\vspace*{1cm}
\begin{center}
\huge\textbf{Ehrenwörtliche Erklärung}\\
\end{center}
\vspace*{1cm}
\normalsize
Wir erklären hiermit ehrenwörtlich,

\begin{enumerate}
	\vspace{1cm}
	\item dass wir unsere Belegarbeit mit dem Thema:\\
	
	\textbf{\titel }\\

	ohne fremde Hilfe angefertigt haben,
	\item dass wir die Übernahme wörtlicher Zitate aus der Literatur sowie die\\ 		  
	Verwendung der Gedanken anderer Autoren an den entsprechenden\\
	Stellen innerhalb der Arbeit gekennzeichnet haben und
	\item dass wir unsere Belegarbeit bei keiner anderen Prüfung vorgelegt haben.\\[1,5cm]
\end{enumerate}
Wir sind uns bewusst, dass eine falsche Erklärung rechtliche Folgen haben wird.\\[1,5cm]
		
\vfill

Glauchau, \abgabedatum\newline\noindent\rule{0.35\columnwidth}{0.4pt}\hspace{0.05\columnwidth}\rule{0.6\columnwidth}{0.4pt}\\
Ort, Datum\hspace{0.27\columnwidth}Unterschriften

{\footnotesize Dies ist eine zur Nutzung mit mehreren Autoren und \LaTeX\ angepasste Version der in \href{https://www.ba-glauchau.de/fileadmin/glauchau/waehrend-des-studium/dokumente/pruefungen/4BA-F.207_Hinweise_zur_Anfertigung_wissenschaftlicher_Arbeiten.pdf}{Anhang 4 der Hinweise zur Anfertigung wissenschaftlicher Arbeiten an der Staatlichen Studienakademie Glauchau vorgegebenen ehrenwörtlichen Erklärung.}}


\newpage
\section{Zustimmung Plagiatsprüfung}

\vspace*{2mm}

\begin{minipage}{0.5\columnwidth}
\includesvg[width=\columnwidth]{bilder/ba_glauchau_logo.svg}
%alternativ mit PNG Logo, falls Inkscape nicht installiert werden, bzw. nicht der PATH-Variable hinzugefügt werden soll
%MERKE: die SVG-Version sieht im Druck deutlich besser aus
%\includegraphics[width=\columnwidth]{bilder/ba-gc-logo.png}
\end{minipage}
\begin{minipage}{0.45\columnwidth}
\begin{flushright}
{\small nach 4BA-F.219\\}
\end{flushright}
\end{minipage}
\vspace*{2mm}

\begin{center}\textbf{\huge{Erklärung zur Prüfung wissenschaftlicher Arbeiten}}\end{center}

Die Bewertung wissenschaftlicher Arbeiten erfordert die Prüfung auf Plagiate. Die hierzu von der Staatlichen Studienakademie Glauchau eingesetzte Prüfungskommission nutzt sowohl eigene Software als auch diesbezügliche Leistungen von Drittanbietern. Dies erfolgt gemäß \href{https://www.revosax.sachsen.de/vorschrift/1672-Saechsisches-Datenschutzgesetz#p7}{§ 7 des Gesetzes zum Schutz der informationellen Selbstbestimmung im Freistaat Sachsen (Sächsisches Datenschutzgesetz - SächsDSG)} vom 25. August 2003 (Rechtsbereinigt mit Stand vom 31. Juli 2011) im Sinne einer Datenverarbeitung im Auftrag.

Die Studierenden bevollmächtigen die Mitglieder der Prüfungskommission hiermit zur Inanspruchnahme o. g. Dienste. In begründeten Ausnahmefällen kann der Datenschutzbeauftragte der Berufsakademie Sachsen sowohl von den Verfassern der wissenschaftlichen Arbeit als auch von der Prüfungskommission in den Entscheidungsprozess einbezogen werden.

\arrayrulewidth=0.5pt

\begin{table}[H]
\centering
\newcolumntype{Y}{>{\centering\arraybackslash}X}
\begin{tabularx}{\columnwidth}{|X|Y|Y|Y|}
\hline
Namen: & \autoreins & \autorzwei & \autordrei \\
\hline
Matrikelnummern: & \matnumeins & \matnumzwei & \matnumdrei \\
\hline
Studiengang: & \multicolumn{3}{c|}{\studiengang}\\
\hline
Titel der Arbeit: & \multicolumn{3}{c|}{Schaltung eines binären}\\
& \multicolumn{3}{c|}{3-Bit-Additions-Subtraktionswerks}\\
\hline
Datum: & \multicolumn{3}{c|}{\abgabedatum}\\
\hline
Unterschriften: & & & \\
&&&\\
\hline
\end{tabularx}
\end{table}

\vfill

{\footnotesize Dies ist eine zur Nutzung mit mehreren Autoren und \LaTeX\ angepasste Version der in \href{https://www.ba-glauchau.de/fileadmin/glauchau/waehrend-des-studium/dokumente/pruefungen/4BA-F.207_Hinweise_zur_Anfertigung_wissenschaftlicher_Arbeiten.pdf}{Anhang 6 der Hinweise zur Anfertigung wissenschaftlicher Arbeiten an der Staatlichen Studienakademie Glauchau vorgegebenen Zustimmung zur Plagiatsprüfung.}}

%  Abstract und Freigabeerklärung zur Bachelorthesis
%! Falls nötig (z.B. bei Diplomarbeit) innerhalb der Datei Wortlaut anpassen, sowie Freigabeerklärung anpassen (erfordert Format Name, Vorname usw.)


%\cleardoublepage
\section{Abstract zur Bachelorthesis}
    % Alternativ zu Vektorgrafik-Logos (SVG) werden Befehle zur Nutzung von Pixelgrafik-Logos (z.B. PNG) bereitgestellt,
    % falls Inkscape nicht installiert werden, bzw. nicht der PATH-Variable hinzugefügt werden soll.
    %! Die SVG-Version sieht im Druck deutlich besser aus.
    \begin{minipage}{0.5\columnwidth}
        \includesvg[width=\columnwidth]{bilder/vorlage/ba-gc-logo}
        %\includegraphics[width=\columnwidth]{bilder/vorlage/ba-gc-logo}
    \end{minipage}
    \begin{minipage}{0.45\columnwidth}
        \begin{flushright}
            {\small nach 4BA-F.258\\}
            %! hier Firmenlogo aktivieren!
            %\includesvg[height=10mm,inkscapelatex=false]{bilder/firmenlogo}
            %\includegraphics[height=10mm]{bilder/firmenlogo}
        \end{flushright}
    \end{minipage}
    \par\noindent\rule{\columnwidth}{.5pt}

    \textbf{\Large{Abstract zur Bachelorthesis}}
    \begin{table}[H]
        \centering
        \fontsize{14pt}{15pt}
        \begin{tabularx}{\columnwidth}{p{3.3cm} X}
            \textbf{Studiengang:}               & \studiengang \\
            \textbf{Name:}                      & \autoreins \\
            \textbf{Thema:}                     & \titel \\
            \textbf{Jahr:}                      & \jahr \\
            \textbf{Betreuer:}                  & \begin{tabular}{@{}ll@{}}
                                                      \betreuereins&\small(\institutioneins)\normalsize\\
                                                      \betreuerzwei&\small(\institutionzwei)\normalsize\\
                                                  \end{tabular}\\
        \end{tabularx}
    \end{table}
    \fbox{
        \centering
        \parbox{0.95\columnwidth}{
            \textbf{Ziel}\\
            Lorem ipsum dolor sit amet, consetetur sadipscing elitr, sed diam nonumy eirmod tempor invidunt ut labore et dolore magna aliquyam erat, sed diam voluptua. At vero eos et accusam et justo duo dolores et ea rebum.\\

            \textbf{Methodik}\\
            Lorem ipsum dolor sit amet, consetetur sadipscing elitr, sed diam nonumy eirmod tempor invidunt ut labore et dolore magna aliquyam erat, sed diam voluptua. At vero eos et accusam et justo duo dolores et ea rebum.\\

            \textbf{Ergebnisse}\\
            Lorem ipsum dolor sit amet, consetetur sadipscing elitr, sed diam nonumy eirmod tempor invidunt ut labore et dolore magna aliquyam erat, sed diam voluptua. At vero eos et accusam et justo duo dolores et ea rebum. Stet clita kasd gubergren, no sea takimata sanctus est Lorem ipsum dolor sit amet.\\

            \textbf{Schlussfolgerung}\\
            Lorem ipsum dolor sit amet, consetetur sadipscing elitr, sed diam nonumy eirmod tempor invidunt ut labore et dolore magna aliquyam erat.\\

            \textbf{Schlüsselwörter}\\
            Lorem, ipsum, dolor, sit, amet
        }
    }
    \clearpage
\section{Erklärung Freigabe/Sperre der Bachelorthesis}
    \begin{flushright}
        {\small nach 4BA-F.300\\}
    \end{flushright}
    \begin{center}
        \Large\textbf{Freigabeerklärung}\\
    \end{center}
    \normalsize

    Hiermit erklären wir uns
    %nicht
    einverstanden, dass die Bachelorthesis des Studierenden\\

    Name, Vorname: Mustermensch, Maximilian(e)\hfill Studiengang: \studiengang

    zur öffentlichen Einsichtnahme durch den Dokumentenserver der Bibliothek der Staatlichen Studienakademie Glauchau bereitgestellt wird.

    Thema der Arbeit:\\

    \large\titel\\\normalsize

    \noindent\rule{0.3\columnwidth}{0.4pt}\\
    Ort, Datum

    \vspace*{2cm}
    \noindent\rule{0.5\columnwidth}{0.4pt}\\
    Stempel, Unterschrift des Praxispartners\\

    Arbeit zur Veröffentlichung freigegeben:\hfill ja$\quad\square$\hfill nein$\quad\square$

    \vfill
    \begin{minipage}{0.5\columnwidth}
        \noindent\rule{0.5\columnwidth}{0.4pt}\\
        Datum
    \end{minipage}
    \begin{minipage}{0.45\columnwidth}
        \noindent\rule{0.8\columnwidth}{0.4pt}\\
        Unterschrift Leiter/in d. Studiengangs
    \end{minipage}

